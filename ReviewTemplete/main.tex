\documentclass{article}
\usepackage[utf8]{inputenc}
\usepackage{amsmath}
\usepackage{amssymb}

\newcommand{\beginquote}{``}
\newcommand{\nc}{\newcommand}
\nc{\cG}{\mathcal{G}}
\nc{\cV}{\mathcal{V}}
\nc{\cA}{\mathcal{A}}
\nc{\cB}{\mathcal{B}}
\begin{document}

\title{Review of "Solving the Multi-Agent Pathfinding Problem with Time-Expanded Networks"}
\author{Reviewer \#[Your Reviewer Number]}
\date{\today}

\maketitle

\section*{Summary}

The paper preseents a novel exact algorithm for the Multi-Agent Pathfinding Problem (MAPF) using Time-Expanded Networks (TENs).
The algorithm is initialized with a Time-Expanded-Graph only containing arcs and vertices contained in minimal distance paths (with or without conflits) for all agents.
At each iteration if a feasible solution is found, with the cost equal to the lower bound, then the algorithm has found an optimal solution. 
Otherwise the graph is expanded with the vertices and arcs corresponding to origin-demand paths one unit longer than the previous iteration and the lower bounds are updated.
The authors prove that the iteratevely defined lower bounds are lower bounds for the optimal solution.

\section*{Minor Comments}

\begin{enumerate}

   

\item \textbf{[Page 5, Example 1]} When defining the actions \(L, R, U, D\) (left, right, up, down), it is currently unclear what these functions return when a suitable neighbor does not exist. 
    For instance, if a vertex \(i \in V\) has no left neighbor, \(L(i)\) would be undefined. 
    Consider adding a clarifying statement such as:
    \emph{\beginquote If no suitable neighbor exists for a given action \(L,R,U\) and \(D\), that action leaves the vertex unchanged (e.g., \(L(i) = i\) in the absence of a left neighbor).''}


\item \textbf{[Page 5, Example 1, Plan decriptions in Examples]}:  When illustrating a plan (e.g., \(\pi_{green} = (L, L, U)\)), listing out each intermediate value 
    (\(\pi_{green}[0] = 6\), \(\pi_{green}[1] = 5\), \(\pi_{green}[2] = 4\), etc.) 
    can be overly detailed and may not substantially aid the reader's understanding. 
    For clarity and conciseness, consider omitting the explicit enumeration of all vertex positions 
    and focus instead on the general sequence of actions.

\item \textbf{[Page 6, Section 3]} 
    When defining \(\tau(i,j)\), the meaning of a \beginquote conflict-free'' path between two vertices is not fully clear. 
    The absence of vertex or edge conflicts is a property concerning the plans of all agents collectively, 
    rather than just a single (agent’s) path. It is also unclear how \(\tau(i,j)\) differs from the standard quickest path between two vertices in a weighed graph. 
    We suggest clarifying this distinction by providing an explicit definition of\beginquote conflict-free in this context, 
    and explaining how and if \(\tau(i,j)\) is computed differently from a straightforward quickest-path distance.

    
    \item \textbf{[Page 8, Section 4]}: There is a small typo in the sixth line of this section: "To easy" should be corrected to "To ease".


\item \textbf{[Page 9-10, Section 4]} 
    It is not entirely clear how \(\cG = (\cV, \cA)\) is initialized. 
    Specifically, if \(\cV\) is defined as the union of the sets \(\mathcal{V}_h\) for each agent when \(q=0\), 
    the definition of \(\cV\) becomes circular because \(\cV_h\) is itself defined as a subset of \(\cV\). 
    To resolve this, we suggest either introducing a separate notation 
    for the graph updated during the various iterations
    (e.g., \(\overline{\cG} = (\overline{\cV}, \overline{\cA})\)) 
    or using iteration-specific graphs 
    (e.g., \(\cG^q = (\cV^q, \cA^q)\) to highlight the iteration index), 
    and then defining \(\cG^q\) in terms of \({\cG}\). 
    This approach would clearly distinguish the initial construction from subsequent expansions.

\item \textbf{[Page 9-10, Section 4]}
    Continuing the above point, when defining \(\cG^q\) for \(q=0\), 
    it should be made explicit whether its vertex set is indeed the union of all \(\mathcal{V}_h\). 

\item \textbf{[Page 10, Section 4]} 
    The sentence \emph{“Before the first iteration (\(q=0\)), \(\mathcal{B}\) is empty”} can be omitted, 
    as \(\mathcal{B}\) is defined only for \(q \in \mathbb{N}\) and, for \(q=0\), we  have 
    \(\mathcal{B} = \mathcal{V}^0\). 
    
\item \textbf{[Inconsistent Figure Labels]}: Some figure labels end with a full stop, while others do not.


\end{enumerate}


\section*{Suggestions for Improvement}

For the proof of Theorem 3, for the first \(\underline{y}^{(0)}- \min_{h \in N}\xi_h\) iterations the graph \(\mathcal{G}\) could be expanded by a smaller number of nodes making the iterations quicker.
Each agent graph \(\mathcal{V}_h\) can be expanded independently. Thus, we can only expand the graphs \(\mathcal{V}_h\) only for those agents where \(\xi_h < \underline{y}^{(0)}\), for exactly \( \underline{y}^{(0)} - \xi_h\) iterations.
After the first \(\underline{y}^{(0)}- \min_{h \in N}\xi_h\) iterations, \(\mathcal{V}_h\) contains every plan  \(\pi_h\) such that \[|\pi_h| \leq  \xi_h + (\underline{y}^{(0)}- \xi_h) = \underline{y}^{(0)}.\]
Thus, \(\mathcal{G}\) contains every sequence of \(n\) plans with a makespan \(y^{(0)} \leq \underline{y}^{(0)}\). Then at each iteration \(q > \underline{y}^{(0)} - \min_{h \in N}\xi_h\), every \(\mathcal{V}_h\) can be expanded for every agent \(h\), andby the same reasoning we have 
that \(\mathcal{G}\) contains every sequence of \(n\) plans with a makespan \(y^{(0)} \leq \underline{y}^{(q)}\) so Theorem 3 still holds with the same definition of the lower bound \(\underline{y}_h\) but adding fewer nodes at each iteration.


\section*{Recommendation}

State your overall recommendation (e.g., accept, minor revision, major revision, reject) and briefly justify your decision.

\end{document}
