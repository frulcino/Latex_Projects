\documentclass{article}
\usepackage[utf8]{inputenc}

\begin{document}

\title{Review of "Solving the Multi-Agent Pathfinding Problem with Time-Expanded Networks"}
\author{Reviewer \#[Your Reviewer Number]}
\date{\today}

\maketitle

\section*{Summary}

The paper preseents a novel exact algorithm for the Multi-Agent Pathfinding Problem (MAPF) using Time-Expanded Networks (TENs).
The algorithm is initialized with a Time-Expanded-Graph only containing arcs and vertices contained in minimal distance paths (with or without conflits) for all agents.
At each iteration if a feasible solution is found, with the cost equal to the lower bound, then the algorithm has found an optimal solution. 
Otherwise the graph is expanded with the vertices and arcs corresponding to origin-demand paths one unit longer than the previous iteration and the lower bounds are updated.
The authors prove that the iteratevely defined lower bounds are lower bounds for the optimal solution.

\section*{Minor Comments}

\begin{enumerate}

    \item \textbf{[Page 8, Section 4]}: There is a small typo in the sixth line of this section: "To easy" should be corrected to "To ease".

    \item \textbf{[Inconsistent Figure Labels]}: Some figure labels end with a full stop, while others do not.

    
   
\end{enumerate}


\section*{Suggestions for Improvement}

For the proof of Theorem 3, the first \(\underline{y}^{(0)}- \min_{h \in N}\xi_h\) iterations the graph \(\mathcal{G}\) could be expanded by a smaller number of nodes making the iterations quicker.
Each agent graph \(\mathcal{V}_h\) can be expanded independently. Thus, we can only expand the graphs \(\mathcal{V}_h\) only for those agents where \(\xi_h < \underline{y}^{(0)}\), for exactly \( \underline{y}^{(0)} - \xi_h\) iterations.
After the first \(\underline{y}^{(0)}- \min_{h \in N}\xi_h\) iterations, \(\mathcal{V}_h\) contains every plan  \(\pi_h\) such that \[|\pi_h| \leq  \xi_h + (\underline{y}^{(0)}- \xi_h) = \underline{y}^{(0)}.\]
Thus, \(\mathcal{G}\) contains every sequence of \(n\) plans with a makespan \(y^{(0)} \leq \underline{y}^{(0)}\). Then at each iteration \(q > \underline{y}^{(0)} - \min_{h \in N}\xi_h\), every \(\mathcal{V}_h\) can be expanded for every agent \(h\), andby the same reasoning we have 
that \(\mathcal{G}\) contains every sequence of \(n\) plans with a makespan \(y^{(0)} \leq \underline{y}^{(q)}\) so Theorem 3 still holds with the same definition of the lower bound \(underline{y}_h\) but adding fewer nodes at each iteration.


\section*{Recommendation}

State your overall recommendation (e.g., accept, minor revision, major revision, reject) and briefly justify your decision.

\end{document}
