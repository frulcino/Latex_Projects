\documentclass[a4paper,11pt]{report}
\usepackage[english]{babel}
%\usepackage{mathmod}
\usepackage[]{ifthen}
\usepackage[]{hyperref}
\usepackage[]{xcolor}
\usepackage[left=3cm,right=3cm,top=3cm,bottom=3cm]{geometry}
\usepackage[]{nth}

\pagenumbering{gobble}
\hypersetup{
  colorlinks=true,
  linkcolor=blue!50!red,
  urlcolor=green!70!black
}

\thispagestyle{empty}

\newcommand{\corso}[6]{\noindent \textbf{#1}, \\ 
\textit{Held by} #2, at the University of #3. \\
#4 hours\ifthenelse{#5 = 1}{ -- with final exam.}{.} }



\begin{document}
\begin{center}
    \Large
    PhD in Computational Mathematics, Learning, and Data Science  \\ 3\textsuperscript{th} cycle -- First Year Final Report\

    \vspace*{5mm}

    \small 
    Gabor Riccardi
\end{center}

\section*{Course attendance}
\begin{itemize}
    \item \corso{Nonlinear approximation}{Prof. Carlo Marcati, Prof.  Pietro Zanotti}{Pavia}{28}{1}
    
    \item \corso{TeX/LaTeX}{Jonathan Franceschi}{Pavia}{12}{1} -- Transversal course.
    
    \item \corso{Computation Optimal Transport}{Prof. Stefano Gualandi}{Pavia}{19}{0} 
     
    \item \corso{Integer Programming and Combinatorial Optimization}{Prof. Sophie Huiberts, Prof. Neil Olver, Prof. Vera Traub at the Institute of Computer Science, University of Wrocław}{Breslavia}{12}{0} 

\end{itemize}

\section*{Projects and Research Activity}
\begin{itemize}
    \item Internship at the Joint Research Centre in Ispra in the project "Resilience assessment of energy systems" for a duration of 5 months in Ispra (VA, Italy). On the generation of scenarios for contingencies in the energy grid and development of a decomposition algorithm for the capacity expansion of the grid.  
    \item Presentation of the work "A decomposition algorithm for the capacity expansion of the grid" at the conference \href{https://www.unical.it/ayw2024/programme/}{AIROYoung}, at the workshop \href{https://hexagon.deib.polimi.it/workshop/}{HEXAGON} and at the poster session at \href{https://sites.google.com/universitadipavia.it/compmat-spring-workshop/poster-session}{COMPMAT workshop} and at the conference \href{https://ipco2024.ii.uni.wroc.pl/}{IPCO}.
    \item Presentation of the joint work "Projective metrics in coding theory" done in collaboration with the PhD student Sauerbier Couvée, for the  project "Reading seminars" supervised by Professor Stefano Gualandi.
    \item Worked on the project "A linear relaxation of the ACOPF based on loop constraints" with PhD student Ambrogio Maria Bernardelli presented at the workshop \href{https://www.hexagon.deib.polimi.it/workshop/}{HEXAGON}.
    \item Participation in the “16th AIMMS-MOPTA Optimization Modeling Competition” with master student Bianca Urso and with Prof. Stefano Gualandi. Presentation of our work "Modeling a Fully Renewable Energy Grid with Hydrogen Storage: An Iterative Time Aggregation Approach" at the conference \href{https://coral.ise.lehigh.edu/~mopta/}{MOPTA 2024} at Lehigh University in
    Bethlehem, PA. Third place out of 16 teams.
\end{itemize}


\newpage
Pavia -- \today
\begin{flushleft}
    \leftskip=9.5cm	
    \emph{Gabor Riccardi}
\end{flushleft}
\vspace*{0.75cm}

\begin{flushleft}
    \leftskip=9.5cm	
    \rule{50mm}{0.25mm}
\end{flushleft}

\vspace{1.5cm}
	
\begin{flushleft}
    \leftskip=9.5cm	
    Tutor: \emph{Prof. Stefano Gualandi}
\end{flushleft}

\vspace*{0.75cm}
\begin{flushleft}
    \leftskip=9.5cm	
    \rule{50mm}{0.25mm}
\end{flushleft}


\vspace{1cm}

\end{document}