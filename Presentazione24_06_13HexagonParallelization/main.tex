\documentclass[11pt, aspectratio=169]{beamer}

\usepackage[utf8]{inputenc}
\usepackage{csquotes}
\usepackage{float}
\usepackage[english]{babel}
\usepackage{tikz-cd}
\usepackage{amsthm}
\usepackage{bbm}
\usepackage{enumitem}
\usepackage{tikz}
\usepackage{amsmath}
\usepackage{amsfonts}
\usepackage{amssymb}
\usepackage{mathtools}
\usepackage{graphicx}
%\usepackage{rotating}
%\usepackage{setspace}
\usepackage{color}
%\usepackage{fancyhdr}
%\usepackage{ragged2e}
\usepackage{appendix}
%\usepackage{tabularx}
%\usepackage{multirow}
%\usepackage{booktabs}
\usepackage{xfrac}
\usepackage{xcolor}
\usepackage{pgfplots}
%\usepackage{url}
%\usepackage{emptypage}
%\usepackage{wrapfig}
\usepackage{dsfont}
%\usepackage{makecell}
\usepackage{bm}
%\usepackage{csquotes}
\usepackage{quiver}


\pgfplotsset{compat=1.18}

%TODOS:
%todo: correggere secondo workflow
%todo: think about rays in dual set
%todo: write better convergence 4
%correct conclusiond ad future directions
%Operators
\DeclareMathOperator{\CVar}{CVar}
\DeclareMathOperator{\Span}{Span}
\DeclareMathOperator{\fq}{q}
\DeclareMathOperator{\fb}{b}

%shortcuts
\newcommand{\nc}{\newcommand} 
\nc{\cH}{{\mathcal H}}
\nc{\cR}{{\mathcal R}}
\nc{\cA}{{\mathcal A}}
\nc{\cG}{{\mathcal G}}
\nc{\cC}{{\mathcal C}}
\nc{\cD}{{\mathcal D}}
\nc{\cO}{{\mathcal O}}
\nc{\cI}{{\mathcal I}}
\nc{\cB}{{\mathcal B}}
\nc{\cY}{{\mathcal Y}}
\nc{\cK}{{\mathcal K}} 
\nc{\cX}{{\mathcal X}}
\nc{\cS}{{\mathcal S}}
\nc{\cE}{{\mathcal E}}
\nc{\cF}{{\mathcal F}}
\nc{\cZ}{{\mathcal Z}}
\nc{\cQ}{{\mathcal Q}}
\nc{\cN}{{\mathcal N}}
\nc{\cP}{{\mathcal P}}
\nc{\cL}{{\mathcal L}}
\nc{\cM}{{\mathcal M}}
\nc{\cT}{{\mathcal T}}
\nc{\cW}{{\mathcal W}}
\nc{\cU}{{\mathcal U}}
\nc{\cJ}{{\mathcal J}}
\nc{\cV}{{\mathcal V}}
\nc{\bH}{{\mathbb H}}
\nc{\bA}{{\mathbb A}}
\nc{\bG}{{\mathbb G}}
\nc{\bC}{{\mathbb C}}
\nc{\bO}{{\mathbb O}}
\nc{\bI}{{\mathbb I}}
\nc{\bB}{{\mathbb B}}
\nc{\bY}{{\mathbb Y}}
\nc{\bK}{{\mathbb K}} 
\nc{\bX}{{\mathbb X}}
\nc{\bS}{{\mathbb S}}
\nc{\bE}{{\mathbb E}}
\nc{\bF}{{\mathbb F}}
\nc{\bZ}{{\mathbb Z}}
\nc{\bQ}{{\mathbb Q}}
\nc{\bN}{{\mathbb N}}
\nc{\bP}{{\mathbb P}}
\nc{\bL}{{\mathbb L}}
\nc{\bM}{{\mathbb M}}
\nc{\bT}{{\mathbb T}}
\nc{\bW}{{\mathbb W}}
\nc{\bU}{{\mathbb U}}
\nc{\bD}{{\mathbb D}}
\nc{\bJ}{{\mathbb J}}
\nc{\bV}{{\mathbb V}}
\nc{\bR}{{\mathbb R}}

\nc{\boB}{{\mathbf{B}}}
\nc{\boL}{{\mathbf{L}}}
\nc{\boG}{{\mathbf{G}}}


\nc{\tV}{{\Tilde{{V}}}}
\nc{\tI}{{\Tilde{{I}}}}
\nc{\tY}{{\Tilde{{Y}}}}
\nc{\tS}{{\Tilde{{S}}}}

\nc{\fr}{{\rightarrow}}
\nc{\co}{{\nabla}}

\newcommand{\la}{\; \longrightarrow \;}
\nc{\cu}{{\barline{\nabla}}}


\usepackage[
backend=biber,
style=alphabetic,
sorting=nty
, maxbibnames=99]{biblatex}

\addbibresource{sample.bib}




\usepackage{ragged2e} % giustifica
\justifying
\setbeamertemplate{caption}{\insertcaption}
\setbeamercovered{invisible}
%\setbeamertemplate{footline}[frame number]
\usepackage{tikz}
\usetikzlibrary{arrows,%
                shapes,positioning}
                
                \definecolor{blendedblue}{rgb}{0.2,0.2,0.7}
        
\usetikzlibrary{%
                petri,%
                topaths}%
%\usepackage{tikz-berge}


\tikzstyle{startstop} = [rectangle, rounded corners, minimum width=3cm, minimum height=1cm,text centered, draw=black, fill=red!30]
\tikzstyle{process} = [rectangle, minimum width=1cm, minimum height=1cm, text centered, draw=black, fill=orange!30]
\tikzstyle{decision} = [rectangle, minimum width=1cm, minimum height=1cm, text centered, draw=black, fill=green!30]
\tikzstyle{arrow} = [thick,->,>=stealth]


\usepackage{ifpdf} 
\ifpdf% 
        \usepackage{pdftricks} 

        \begin{psinputs} 
            \usepackage{pstricks} 
            \usepackage{pstricks-add} 
            \usepackage{pst-plot} 
            \usepackage{pst-text,pst-node,pst-tree} 
        \end{psinputs} 
\else 
        \usepackage{pstricks} 
        \usepackage{pstricks-add} 
        \usepackage{pst-plot} 
        \usepackage{pst-text,pst-node,pst-tree} 

\fi 

\usepackage{pstricks}

\makeatletter
\setbeamertemplate{footline}
{
  \leavevmode%
  \hbox{%
  \begin{beamercolorbox}[wd=.333333\paperwidth,ht=2.25ex,dp=1ex,left]{author in head/foot}%
    \usebeamerfont{author in head/foot}\hspace*{2ex}\insertauthor
  \end{beamercolorbox}%
  \begin{beamercolorbox}[wd=.333333\paperwidth,ht=2.25ex,dp=1ex,center]{title in head/foot}%
    \usebeamerfont{title in head/foot}\insertsubsection
  \end{beamercolorbox}%
  \begin{beamercolorbox}[wd=.333333\paperwidth,ht=2.25ex,dp=1ex,right]{date in head/foot}%
    \usebeamerfont{date in head/foot}\insertshortdate{}\hspace*{2em}
    \insertframenumber{} / \inserttotalframenumber\hspace*{2ex} 
  \end{beamercolorbox}}%
  \vskip0pt%
}
\makeatother
\setbeamertemplate{navigation symbols}{}

\usepackage{color}
\definecolor{deepblue}{rgb}{0.3,0.3,0.9}
\definecolor{deepred}{rgb}{0.6,0,0}
\definecolor{deepgreen}{rgb}{0,0.5,0}

\setlist[itemize]{label=\textcolor{deepblue}{\rule[0.5ex]{1ex}{1ex}}}
\usepackage{listings}

\definecolor{codegreen}{rgb}{0,0.6,0}
\definecolor{codegray}{rgb}{0.5,0.5,0.5}
\definecolor{codepurple}{rgb}{0.58,0,0.82}
\definecolor{backcolour}{rgb}{0.95,0.95,0.92}

\lstdefinestyle{mystyle}{
    backgroundcolor=\color{backcolour},   
    commentstyle=\color{codegreen},
    keywordstyle=\color{magenta},
    numberstyle=\tiny\color{codegray},
    stringstyle=\color{codepurple},
    basicstyle=\ttfamily\footnotesize,
    breakatwhitespace=false,         
    breaklines=true,                 
    captionpos=b,                    
    keepspaces=true,                 
    numbers=left,                    
    numbersep=5pt,                  
    showspaces=false,                
    showstringspaces=false,
    showtabs=false,                  
    tabsize=2
}

\lstset{style=mystyle}


\newtheorem{prop}[theorem]{Proposition}
\newtheorem{defi}[theorem]{Definition}
\newtheorem{oss}[theorem]{Observation}
\newtheorem{theo}[theorem]{Theorem}
\newtheorem{cor}[theorem]{Corollary}
\newtheorem{assumption}[theorem]{Assumption}
\renewcommand*{\bibfont}{\footnotesize}
\title{A Parallelization Algorithm for Adequacy Assessment of the Electrical Grid}
\author{Gabor Riccardi}
\date{19/06/24}
\institute{University of Pavia}

\begin{document}

\begin{frame}[plain]
  \begin{minipage}{0.3\textwidth}
    \centering
    \includegraphics[width=.6\linewidth]{unipv.png}\\[\baselineskip] % Adjust space as needed
    \includegraphics[width=.8\linewidth]{EC-JRC-logo.png}
  \end{minipage}%
  \begin{minipage}{0.05\textwidth}
    \tikz[remember picture, overlay] \draw[yellow, thick] (0,0) -- (0,-\textheight);
  \end{minipage}%
  \begin{minipage}{0.65\textwidth}
    \maketitle
    \centering
    %\includegraphics[width=.6\linewidth]{AIROY.png}
  \end{minipage}
\end{frame}

% \section{Description of the Adequacy Assment problems}
% \begin{frame}{Power Grid Optimization problems}
%     \begin{figure}
%         \centering
%         \includegraphics[width=0.8\textwidth]{photo_2023-09-26_14-27-16.jpg}
%         \label{fig:enter-label}
%     \end{figure}
% \end{frame}

\begin{frame}{Table of Contents}
\tableofcontents
\end{frame}



\begin{frame}{Adequacy Assessment of the Electrical Grid}
  \begin{itemize}
    \item Measures the ability of the electric power system to react to adverse uncertain condition. \pause
    \item Member States wishing to introduce capacity mechanisms can do so if an adequacy concern is identified in the ERAA study, a pan-European adequacy assessment for up to 10 years ahead. \pause
    \item Due to the scale of the ERAA study, ERAA 2022 considered a reduced stochastic problem with three scenarios. \pause
    \item To adress this issue \cite{DecompAlg}, Daniel A'vila introduced a decomposition algorithm based on subgradient approximations. \pause
  \end{itemize}
\end{frame}

\section{Stochastic Capacity Expansion Problem (CEP)}

\begin{frame}{Stochastic Capacity Expansion Problem (CEP)}

  We formulate the Stochastic Capacity Expansion Problem as a two-stage stochastic program. \pause
  \begin{align*}
    \min_{x} \; & c'x + \bE_{\omega}\left[\cV(x,\omega)\right] \\  \tag{CEP}
    s.t. \;     & 0 \leq x_{n,g} \leq X_{n,g}
  \end{align*}
  \begin{itemize}
    \pause
    \item The first stage determines the capacity expansion \(x_{n,g}\) for each generator \(g \in \cG \) \pause
    \item The second stage solves the Economic Dispatch (ED). \pause
  \end{itemize}

  Where \(\cV(x,\omega)\) is the solution to (ED) in function of the expanded capacities \(x\) and the scenario \(\omega\).
\end{frame}

\begin{frame}{Economic Dispatch (ED) variables}
  The scenarios \( \omega \in \Omega\) comprise of the realization of the following variables:\pause\\
  \begin{tabular}{@{}c p{0.7\textwidth}@{}}
    \quad \includegraphics[height=1em]{smolimages/solar-panels.png} & Solar power \(\mathbf{\mathcal{PV}_{\omega}}\) (MW)\pause \\
    \quad \includegraphics[height=1em]{smolimages/wind-power.png} & Wind power \(\mathbf{\mathcal{WP}_{\omega}} \) (MW)\pause \\
    \quad \includegraphics[height=1em]{smolimages/power.png} & Loads \(\mathbf{\mathcal{D}_{\omega}}\) (MW)\pause \\
  \end{tabular}
\vspace{1cm} \\
  The optimization variables are:\pause\\

  \begin{tabular}{@{}c p{0.7\textwidth}@{}}
    \quad \includegraphics[height=1em]{smolimages/battery.png} & The power stored at each bus: \(\mathbf{v_{n,t,w}}\) \pause\\
    \quad \includegraphics[height=1em]{smolimages/smart-grid.png} & Power generation: \(\mathbf{p_{\omega}}\) (MW)\pause\\
    \quad \includegraphics[height=1em]{smolimages/power.png} & Power flow through line \(l\): \(\mathbf{f_{n,l,\omega}}\) (MW)\pause\\
    \quad \tikz \fill[yellow] (0,0) rectangle (1ex, 1ex); & Load shedding \(\mathbf{ls_{\omega}}\) (MW)\pause\\
    \quad \tikz \fill[yellow] (0,0) circle (0.5ex); & Spillage  \(\mathbf{s_{\omega}}\) (MW)\pause\\
   
  \end{tabular}
\end{frame}

\begin{frame}{Economic Dispatch (ED) model \; \only<2>{\textcolor{red}{Scary Slide}}}
  
 
  \begin{align}
    \uncover<3-9>{\min_{y} \; & q'y_{\omega}                                                                                                                                            \\}
    \uncover<4-9>{s.t. \;     & p_{n,g,t,{\omega}} + bd_{n,t,{\omega}} + \sum_{l \in \cL(n)}f_{n,l,t,{\omega}} + ls_{n,t,{\omega}} + \mathcal{PV}_{n,t,{\omega}} + \cW_{n,t,{\omega}} = \\}
    \uncover<5-9>{            & \quad \quad =  \cD_{n,t,{\omega}} + s_{nt.{\omega}} + bc_{n,t,{\omega}} \nonumber                                                                      \\}
    \uncover<6-9>{           & v_{n,t,{\omega}} = v_{n,t-1,{\omega}} + BCE \cdot bc_{n,t,{\omega}} - BDE \cdot bd_{n,t,{\omega}} + A_{n,t,{\omega}}                                    \\}
    \uncover<7-9>{                & (v_{n,t,{\omega}}, bc_{n,t,{\omega}}, bd_{n,t,{\omega}}) \leq (BV, BC, BD)                                                                              \\}
    \uncover<8-9>{  & p_{n,g,t,{\omega}} \leq p^{\text{max}}_{n,g} + x_{n,g}                               \nonumber                                                                   \\}
    \uncover<9-9>{             & L^{\text{min}}_{n, l} \leq f_{n,l,t,{\omega}} \leq L^{\text{max}}_{n, l} \nonumber}
  \end{align}

 
\end{frame}

%import parts


\section{Literature Review}

\begin{frame}{Literature Review - 1/2}
  In \cite{DecompAlg}, A'vila et Al, the time horizon is divided  into \(K\) intervals: \\ 
   \pause \[\{0,\ldots, t_{1}\}, \ldots,\{t_{K-1}+1,\ldots,t_K = T\}.\] \pause
  
  Then, for each \(k\), the economic dispatch restricted to the time steps \(T \geq t \geq  t_k\) is considered. \pause \\
  \vspace{0.5cm}
  Let  \(\cV_k(x,v_{t_k},\omega)\) be the corresponding optimal value, it can be defined inductively as: \pause\\

  \begin{alignat}{2}
    \cV_k(x,v_{t_k},\omega) = \min & \Bigl[\, (1)\, \Bigr]_{t = t_k}^{t_{k+1}-1} +  \cV_{k+1}(x,v_{t_{k+1}},\omega) \nonumber \\
    & \text{s.t.} \Bigl[\, (2)-(6)\, \Bigr]_{t_{k-1}+1}^{t_k} \nonumber
  \end{alignat}
  \pause
  Where \(\cV_{K+1} \coloneqq 0\). \pause Note that \(\cV_1(x,\omega) = \cV(x, \omega)\). \pause \\
  

  %this is kind of as saying to the grid, hey you start this storage levels, but must end at this other storage levels

\end{frame}

\begin{frame}{Literature Review - 2/3}
  Since each \(\cV_k\) is peacewise convex in \(x\) and \(v_{t_k}\), it can be approximated by a collection of supporting hyperplanes \(\{\pi^w_{i,k}(x,v_{t_k})\}\) of each \(\cV_k\): \pause

  \begin{alignat}{2}
    \hat{\cV}_k(x,v_{t_k},\omega) = \min & \Bigl[\, (1)\, \Bigr]_{t = t_k}^{t_{k+1}-1} +  \theta_{k+1,\omega} \nonumber \\
    \text{s.t.} & \Bigl[\, (2)-(6)\, \Bigr]_{t_{k-1}+1}^{t_k} \nonumber \\
         & \theta_{k+1,\omega} \geq \pi^w_{i,k}(x,v_{t_{+1}}) \nonumber
  \end{alignat}

  \pause

  Then the relaxed capacity expansion problem, (CEP-A), is define as:
\pause
  \begin{align*}
    \label{CEP-A}
    \min_{x} \; & c'x + \bE_w\left[\hat{\cV}_1(x,w)\right] \\  \tag{CEP-A}
    \text{s.t.} \;     & 0 \leq x_{n,g} \leq X_{n,g}
  \end{align*}
\pause
  This can be solved efficiently with L-shaped or subgradient schemes.
\end{frame}

\begin{frame}{Literature review - 3/3 Algorithm description}

  \resizebox{0.8\textwidth}{!}{
  \begin{tikzpicture}[node distance=1.5cm]
      % Nodes
      \uncover<2-9>{\node (start) [startstop] {\textbf{Initialize}: Provide a lower bound for \(\cV_k\) and an initial trial action \(\hat x^0\)};}
      \uncover<3-9>{\node (pro0) [process, below of=start] {Solve CEP-A using current lower approximation \(\hat{\cV}_1\) and obtaining new trial action \(\hat{x}^i\).};}
      \uncover<4-9>{\node (pro1) [process, below of=pro0] {\textbf{Forward step}: Compute the current approximation \(\hat{\cV}_k(x, v_{t_k},\omega)\) for \(k = 1,\ldots,K\) and \(\forall \omega\)};}
      \uncover<5-9>{\node (pro2) [process, below of=pro1] {\textbf{Backward step}: from the dual of \(\hat{\cV}_{k}\), compute a cut \(\pi_{i+1,k}^{\omega}\), for \(\hat{\cV}_{k-1}\) for \(k = K,\ldots,1\)};}
      \uncover<6-9>{\node (rpro0) [right of=pro0, xshift = 200pt] {};}
      \uncover<7-9>{\node (dec1) [decision, below of=pro2] {Are there any new cuts?};}
      \node (rdec1) [right of=dec1, xshift = 200pt] {};
      \uncover<8-9>{<\node (stop) [startstop, below of=dec1] {\textbf{Stop}: \(\hat{x}^i\) is optimal solutions for CEP};}
      
      % Arrows
      \uncover<2-9>{\draw [arrow] (start) -- (pro0);}
      \uncover<3-9>{\draw [arrow] (pro0) -- (pro1);}
      \uncover<4-9>{\draw [arrow] (pro1) -- (pro2);}
      \uncover<5-9>{\draw [arrow] (pro2) -- (dec1);}
      \uncover<9-9>{\draw [arrow] (2.1,-6) .. controls (11,-6) and (11,-2) .. node[anchor=east] {Yes} (6.8,-1.5);}
      %\draw (dec1) .. controls (rdec1) and (rpro0) .. (prog0);
      \uncover<7-9>{\draw [arrow] (dec1) -- node[anchor=west] {No} (stop);}
  \end{tikzpicture}}
\end{frame}

\begin{frame}{}

  \begin{minipage}[t]{0.45\textwidth}
    \textbf{Definition:} The \emph{hypergraph} associated to a linear programming problem LP, denoted by \(\mathcal{G} = (\mathcal{N}, \mathcal{E})\), is constructed as follows:
    \begin{itemize}
      \item The \emph{nodes} \(\mathcal{N}\) of \(\mathcal{G}\) correspond to the variables of the LP.
      \item The \emph{hyperedges} \(\mathcal{E}\) of \(\mathcal{G}\) correspond to each set of variables that appears together in any constraint of the LP.
    \end{itemize}
  \end{minipage}
  \begin{minipage}[t]{0.45\textwidth}

    \begin{figure}
      \begin{tikzpicture}
        % Node style and position setup
        \foreach \i in {1,...,20}
        {
          \pgfmathparse{int(mod(\i-1,5))} % x position
          \edef\x{\pgfmathresult}
          \pgfmathparse{int((\i-1)/5)} % y position
          \edef\y{\pgfmathresult}
          % Draw a dot instead of a labeled circle
          \fill (1.5*\x,-1.5*\y) circle (2pt) coordinate (x\i); % Place a coordinate for referencing
        }
        
        % Hyperedges as polygons
        \draw[thick, fill=blue!50, fill opacity=0.5] (x1) -- (x4) -- (x11) -- cycle;
        \draw[thick, fill=red!50, fill opacity=0.5] (x16) -- (x5) -- (x8) -- (x10) -- cycle;
        \draw[thick, fill=green!50, fill opacity=0.5] (x3) -- (x6) -- (x13) -- cycle;
        \draw[thick, fill=yellow!50, fill opacity=0.5] (x3) -- (x16) -- (x19) -- (x14) -- (x11) -- cycle;
        \draw[thick, fill=violet!50, fill opacity=0.5] (x4) -- (x17) -- (x20) -- (x12) -- (x15) -- cycle;
        
      \end{tikzpicture}
      \caption{Example of LP hypergraph.}
    \end{figure}
  \end{minipage}
\end{frame}




\begin{frame}{Model relaxation description: Intermediate Economic Dispatches (ED-k)}
  % \textbf{If a hypergraph features a partition of \(\cN\) with sparse interconnections (few edges) between subsets, we can remove these edges to solve each subset independently}. To do this, we fix a priori the variables in the removed edges. We search for the optimal values for the fixed variables through an iteratively tightened linear program. \\
  % \vspace{1cm}
  %For instance, in the case of (ED), we leverage the fact that \textbf{the only constraints connecting variables of different days are the storage constraints}:\\
  
  \begin{minipage}[t]{0.45\textwidth}
    \vspace{0cm}
    \begin{figure}
      \begin{tikzpicture}[plane/.style={draw, thick, fill=blue!20, opacity=0.6}, scale = 0.5]
        % Define the number of planes
        \newcommand\NumPlanes{4}
        % Define the distance between planes
        \newcommand\DistPlanes{1.5}
        % Define a list of colors
        \def\colors{{"red", "blue", "green", "orange", "purple"}}
      
        % Draw the planes
        \foreach \i in {0,...,\NumPlanes} {
          % Calculate y-shift based on the plane index
          \pgfmathsetmacro\Shift{\i*\DistPlanes}
          % Draw a plane
          \draw[plane] (0+\Shift,0+\Shift) rectangle ++(3,2);
          % Add a label at the top of each plane
          \node at (1.5+\Shift,2.2+\Shift) {ED-\pgfmathparse{int(\i+1)}\pgfmathresult};
        }
      
        % Connect the corners of the planes
        \foreach \i in {1,...,\NumPlanes} {
          \pgfmathsetmacro\j{\i-1}
          \pgfmathsetmacro\Shift{\i*\DistPlanes}
          \pgfmathsetmacro\PrevShift{\j*\DistPlanes}
          % Retrieve color from list
          \pgfmathsetmacro\Color{\colors[\i-1]}
          % Draw connecting segments
          \draw[thick, \Color] (\PrevShift, \PrevShift) -- (\Shift, \Shift);
          \draw[thick, \Color] (3+\PrevShift, 2+\PrevShift) -- (3+\Shift, 2+\Shift);
        }
      \end{tikzpicture}
        
      
      \caption{(ED) hypergraph representation.}
    \end{figure}
  \end{minipage}
  \begin{minipage}[t]{0.45\textwidth}
    \pause
    \begin{itemize}
      \item We divide the time horizon into \(K\) intervals: 
      \begin{tabular}{l}
         \(\{t_0 \coloneqq 0 ,\ldots, t_1\}\), \pause\\
         \(\{t_1+1,  \ldots, t_2\}\), \pause\\
          \quad \quad \quad\(\ldots\) \pause  \\
         \(\{t_{K-1}+1, \ldots, t_K = T \}\)\pause
      \end{tabular}
      \item We fix a priori the intermediate storage values \(v_{t_k}\) for \(k = 1,\ldots,K\). \pause
      \item We refer to the (ED) problems restricted to each time interval as \textbf{(ED-k)}  \pause 
      \item The corresponding optimal values are \(\mathbf{\cV_{k}(x,v_{t_{k}},v_{t_{k+1},\omega})}\) \pause %this is kind of as saying to the grid, hey you start this storage levels, but must end at this other storage levels
    \end{itemize}
  \end{minipage}
\end{frame}

\begin{frame}{}
  

  \begin{oss}
    \begin{equation}\label{Divided ED eq}
      \cV(x,\omega) = \min_{\{v_{t_k}\}_{k=1}^K}\sum_{k=0}^{K-1}\cV_{k}(x,v_{t_{k}},v_{t_k+1},\omega)
    \end{equation}
  \end{oss}

\end{frame}

\begin{frame}{Model relaxation description: Lower Approximation of (ED)}
  Since each function \(\cV_k\) is piecewise linear convex in \(x,v_{t_K},v_{t_{K+1}}\), it can be approximated by a collection of supporting hyperplanes \(\{\pi^w_{i,k}(x,v_{t_k},v_{t_{k+1}})\}\) of each \(\cV_k\). \\ \pause
  An approximation of (ED) is given by: \pause
  \begin{align*}
    \hat{\mathcal{V}}(x,\omega) & = \min_{\{v_{t_k}\}_{k=1}^K} \sum_{k=0}^K \hat{\mathcal{V}}_k(x,v_{t_k},v_{t_{k+1}}) =                           \\
                                & = \min_{\{v_{t_k}\}_{k=1}^K} \sum_{k=0}^K \theta_{k}^{\omega} \tag{ISP}                                          \\
                                & \quad \quad \text{s.t.} \quad \theta_k^{\omega} \geq \pi_{i,k}^{\omega}(x,v_{t_k},v_{t_{k+1}}) \quad \forall i,k
  \end{align*}
  We refer to this problem as the \textbf{Intermediate Storage Problem (ISP)} \pause
  \\ (I know, very original)

\end{frame}
\begin{frame}{Model description: Relaxed Capacity Expansion(CEP-R)}
  %Thus by substituting \(\cV\) with \( \hat{\cV} \) in (CEP) we obtain the following relaxation:
  \begin{align*}
    \label{CEP-A}
    \min_{x} \; & c'x + \bE_{\omega}\left[\cV(x,\omega)\right] \\  \tag{CEP}
    \text{s.t.} \;     & 0 \leq x_{n,g} \leq X_{n,g}
  \end{align*}
  \pause
  \begin{align*}
    \label{CEP-R}
    \min_{x} \; & c'x + \bE_w\left[\hat{\cV}(x,w)\right] \\  \tag{CEP-R}
    \text{s.t.} \;     & 0 \leq x_{n,g} \leq X_{n,g}
  \end{align*}
  \pause
  Since calculating \( \hat{\cV} \) is straightforward, solving (CEP-R) can be done efficiently with L-shaped or subgradient schemes.
\end{frame}

\section{Decomposition Algorithm}

\begin{frame}{Algorithm}
  % https://q.uiver.app/#q=WzAsMTAsWzEsMCwiXFx0ZXh0e0lucHV0fSJdLFsxLDEsIlxcdGV4dHtSLUNFUH0iXSxbMSwyLCJJU1AoXFxvbWVnYSkgXFw7IFxcZm9yYWxsIFxcb21lZ2FcXGluXFxPbWVnYSJdLFswLDMsIlxcdGV4dHtFRC0xfSJdLFsxLDMsIlxcdGV4dHtFRC0yfSJdLFsyLDMsIlxcZG90cyJdLFszLDMsIlxcdGV4dHtFRC1LfSJdLFsxLDQsIlxcdGV4dHtDb21wdXRlIG5ldyBjdXRzIGZvciB9IFZfayBcXDsgXFxmb3JhbGwgayJdLFs0LDQsIlxcYnVsbGV0Il0sWzEsNSwiXFxoYXR7eF5pfSBcXHRleHR7IGlzIENFUC1vcHRpbWFsfSJdLFswLDFdLFsxLDIsIlxcaGF0e3h9XmkiXSxbMiwzLCJ2X3t0XzB9LHZfe3RfMX0iLDFdLFsyLDQsInZfe3RfMX0sdl97dF8yfSIsMV0sWzIsNV0sWzIsNiwidl97dF97ay0xfX0sdl97dF9rfSIsMV0sWzMsN10sWzQsNywiXFx0ZXh0e0R1YWwgbXVsdGlwbGllcnN9IiwxXSxbNSw3XSxbNiw3XSxbNyw4LCJcXHRleHR7aWYgbmV3IGN1dHN9IiwxXSxbOCwxLCJcXHRleHR7QWRkIGN1dHN9IiwxLHsiY3VydmUiOjV9XSxbNyw5XV0=
  \begin{center}
    \[\begin{tikzcd}[ampersand replacement=\&]
      \& {\text{Input}} \\
      \& {\text{R-CEP}} \\
      \& {ISP(\omega) \; \forall \omega\in\Omega} \\
      {\text{ED-1}} \& {\text{ED-2}} \& \dots \& {\text{ED-K}} \\
      \& {\text{Compute new cuts for } \mathcal{V}_k \; \forall k} \&\&\& \bullet \\
      \& {\hat{x}^i \text{ is CEP-optimal}}
      \arrow[from=1-2, to=2-2]
      \arrow["{\hat{x}^i}", from=2-2, to=3-2]
      \arrow["{v_{t_0},v_{t_1}}"{description}, from=3-2, to=4-1]
      \arrow["{v_{t_1},v_{t_2}}"{description}, from=3-2, to=4-2]
      \arrow[from=3-2, to=4-3]
      \arrow["{v_{t_{k-1}},v_{t_k}}"{description}, from=3-2, to=4-4]
      \arrow[from=4-1, to=5-2]
      \arrow["{\text{Dual multipliers}}"{description}, from=4-2, to=5-2]
      \arrow[from=4-3, to=5-2]
      \arrow[from=4-4, to=5-2]
      \arrow["{\text{if new cuts}}"{description}, from=5-2, to=5-5]
      \arrow["{\text{Add cuts}}"{description}, curve={height=40pt}, from=5-5, to=2-2]
      \arrow[from=5-2, to=6-2]
    \end{tikzcd}\]
  
    % \begin{tikzpicture}[node distance= 1cm]
    %   \node (in) [process] {Input};
    %   \node (rcep) [process, below left=of in] {R-CEP};
    %   \node (isp) [process, below right=of in] {ISP for all \(\omega \in \Omega\)};
    %   \node (ed1) [process, below left=of isp] {ED-1};
    %   \node (ed2) [process, right=of ed1] {ED-2};
    %   \node (edk) [process, right=of ed2] {ED-K};
    %   \node (cuts) [decision, below=of edk] {Compute new cuts for \(V_k\) for all \(k\)};
    %   \node (optimal) [process, below=of cuts] {Optimal cuts in \(\hat{x}^i\) for \(\hat{x}^i\) is CEP-optimal};
    %   \node (duals) [process, below=of cuts] {Dual multipliers};
    %   \node (add) [process, below=of duals] {Add cuts};
    %   \draw[->] (in) -- (rcep);
    %   \draw[->] (rcep) -- (isp);
    %   \draw[->] (isp) -- (ed1);
    %   \draw[->] (ed1) -- (ed2);
    %   \draw[->] (ed2) -- (edk);
    %   \draw[->] (edk) -- node[above] {\(V_{t_k}, V_{t_{k+1}}\)} (cuts);
    %   \draw[->] (cuts) -- (optimal);
    %   \draw[->] (cuts) -- node[above] {new cuts} (duals);
    %   \draw[->] (duals) -- (add);
    %   \draw[->] (add) -- node[above] {new cuts} (cuts);
    %   \draw[->] (add) -- (optimal);
    % \end{tikzpicture}
  \end{center}
\end{frame}


% \begin{frame}{Algorithm}
%   \includegraphics[width = 0.7\linewidth]{Alg.png}
% \end{frame}
% \begin{frame}{Algorithm description} %maybe we can devide this description in the previous slides
%   INPUT: Provide a lower bound for \(\theta_k^{\omega}\) for \(k=1,\ldots,K\) and \( \omega \in \Omega \) and a trial action \(\hat x^0\)
%   \pause
%   \begin{enumerate}[label = {\arabic*}]
%     \item Warm-Start: Calculate initial approximation for \(\mathcal{V}\) for all \(\omega \in \Omega\) around \(\hat{x}^0\). \pause
%     \item For \(i = 1,\ldots,N\): \pause
%           \begin{enumerate}[label = {2.\arabic*}]
%             \item Solve current relaxation (CEP-R) and obtain new trial action \(\hat x^i\). \pause
%             \item For \(\omega \in \Omega \) (in parallel): \pause
%                   \begin{enumerate}[label = {2.1.\arabic*}]
%                     \item Solve the (ISP) approximation problem \(\hat{\mathcal{\cV}}(\hat{x}^{(i)},\omega)\) and obtain intermediate storage values \(\hat{v}^i_k\) for \(k=1,\ldots,K\).\pause
%                     \item Solve (ED-k) (in parallel) for each time step.\pause
%                     \item Using dual multipliers, compute a supporting hyperplane for \(\cV_k\) around \(\hat{x}^i, \hat{v}^i_k, \hat{v}^i_{k+1}\) for \(k=0,\ldots,K-1\).\pause
%                     \item Add the supporting hyperplanes to the approximation problems (ISP) and(CEP-R) \(\hat{\mathcal{V}}(\hat{x}^{(i)},\omega)\).
%                   \end{enumerate}
%           \end{enumerate}
%   \end{enumerate}
% \end{frame}

% \begin{frame}{Algorithm description}
%   INPUT: Provide a lower bound for \(\theta_k^{\omega}\) for \(k=1,\ldots,K\) and \( \omega \in \Omega \) and a trial action \(\hat x^0\)

%   \begin{tabular}{@{}p{0.95\textwidth}}
%     1. Warm-Start: Calculate initial approximation for \(\mathcal{V}\) for all \(w \in \Omega\) around \(\hat{x}^0\). \\
%     2. For \(i = 1,\ldots,N\): \\
%     \quad 2.1. For \(w \in \Omega \) (in parallel): \\
%     \quad\quad\quad 2.1.1. Solve the (ED) approximation problem \(\hat{\mathcal{V}}(\hat{x}^{(i)},\omega)\) and obtain 
%     \\ \quad\quad\quad\quad  intermediate storage values \(\hat{v}^i_k\) for \( k=1,\ldots,K\). \\
%     \quad\quad\quad 2.1.2. Solve (ED) (in parallel) for each time step \(k = 0,\ldots,K-1\),
%     \\ \quad\quad\quad\quad\(V_k(\hat{x}^i,\hat{v}^i_k, \hat{v}^i_{k+1},\omega)\). \\
%     \quad\quad\quad 2.1.3. Using dual multipliers, compute a supporting hyperplane for \(V_k\) around 
%     \\ \quad\quad\quad\quad\(\hat{x}^i, \hat{v}^i_k, \hat{v}^i_{k+1}\) for \(k=0,\ldots,K-1\). \\
%     \quad\quad\quad 2.1.4. Add the supporting hyperplanes to the approximation problem (CEPR)
%     \\ \quad\quad\quad\quad  \(\hat{\mathcal{V}}(\hat{x}^{(i)},\omega)\). \\

%   \end{tabular}
% \end{frame}


\section{Convergence results}
\begin{frame}{Convergence results - 1/5}
  \begin{itemize}
    \item Since \((CEP-R) \leq (CEP)\) if a \((CEP-R)\) optimal solution is (CEP) feasible then it's also \((CEP)\)-optimal. \pause
  \end{itemize}

  \hfill \\


  \textbf{Remark 1:} It is sufficient to prove that after a finite number of steps \((i)\) of the algorithm we have:
  \begin{equation}
    \hat{\cV}(\hat x^i,\omega) = \cV(\hat x^i,\omega)  \text{ for all }  \omega \in \Omega
  \end{equation}


\end{frame}


\begin{frame}{Convergence results - 2/5}
  \begin{oss}
    After a finite number of iterations no new cuts are found for \(\cV_k\).
  \end{oss}
  \begin{proof}
    \begin{align}
       & \uncover<2->{\#\{p \mid p \text{ is a normal vector of a supporting hyperplane of } \mathcal{V}_k\} \leq \nonumber }\uncover<3->{ \\
       & \#\{\text{dual solutions } p=q'B^{-1} \text{ of (ED-k) for varying } x, v_{t_k}, v_{t_{k+1}}\} \leq \nonumber }\uncover<4->{      \\
       & \#\{\text{basis matrices of (ED-k)}\} < \infty }
    \end{align}
    \pause \pause \pause \pause
    After a finite number of steps: \pause
    \begin{itemize}

      \item new cut: \(\bar c (x,v)= \textcolor{deepblue}{p}'(x,v) + b\)  \pause
      \item an old cut: \(\pi(x,v) = \textcolor{deepblue}{p}'(x,v) + \bar b\)

    \end{itemize}


  \end{proof}
\end{frame}


\begin{frame}[noframenumbering]{Convergence results - 3/5}



  \begin{tikzpicture}[scale=1]
    % Draw axes
    \draw[->] (-1,0) -- (4,0) node[right] {$x$};
    \draw[->] (0,-1) -- (0,3) node[above] {$y$};

    % Define function breakpoints
    \def\xbreak{2}
    \def\ybreak{2}

    % Draw piecewise function
    \draw[thick] (0,2.3) -- (\xbreak,\ybreak);
    \draw[thick] (\xbreak,\ybreak) --  (4,3);

    % Draw tangent line (touching the curve at one point)
    \draw[thick, red] (0,1.5) -- (4,2.5);

    % Mark the touching point
    \filldraw[black] (\xbreak,\ybreak) circle (1pt);

    % Draw parallel line slightly beneath the first one
    \draw[thick, green] (0,1) -- (4,2);

    \draw (3,3) node {$\cV_k$};
    \draw (1,1.90) node[red] {$\pi$};
    \draw (1,1.41) node[green] {$\bar c$};

  \end{tikzpicture}
  \pause
  \\ Since both are supporting hyperplanes it follows that \( b = \bar b\) \\(and therefore \( \bar c\) is not a new cut).
\end{frame}
\begin{frame}{Convergence results - 4/5}
  \begin{oss}
    If after the \(i\)-iteration no new cuts are added for some \(i\) and \(k\) then \(\hat{\cV}_k(\hat x^i, \hat v_{k},\hat v_{k+1}) = \cV_k(\hat x^i, \hat v_{k},\hat v_{k+1}). \) \\ \pause
   
  \end{oss}
  \begin{proof}

    Let \(\bar c_k^{\omega}(x,v_{t_k}) \coloneqq p'(x-\hat x^i, v_{t_k}-\hat v_{t_k}) + \cV_k(\hat x^i,\hat v_{t_k})\) be the new cut found after the \(i\)-th iteration. \\ \pause
    Since \(\bar c\) is not a new cut we have \( \bar c(x,v_{t_k}) \leq \hat\cV_k(x,v_{t_k})\). \pause
    We have thus \[\cV_k(\hat x^i,\hat v_{t_k}) \geq \hat \cV_k(\hat x^i,\hat v_{t_k}) \geq \bar c (\hat x^i,\hat v_{t_k}) = \cV_k(\hat x^i,\hat v_{t_k})\] \pause which concludes the proof.
  \end{proof}

\end{frame}

\begin{frame}{Convergence results - 5/5}
  In conclusion, we have \(\hat{\cV}_k(\hat x^i, v_{t_k},v_{t_{k+1}},\omega) = \cV_k(\hat x^i, v_{t_k},v_{t_{k+1}},\omega)\) for all \(\omega,k\). \pause \\
  Thus \( \hat \cV(\hat x^i,\omega) = \cV(\hat x^i,\omega) \). \pause
  \hfill \\
  \hfill \\
  \begin{prop}
    The algorithm converges after a finite number of iterations and \(\hat x^i\) is an optimal solution for (CEP).
  \end{prop}
\end{frame}

\section{Initial results, Conclusions and Future work}
\begin{frame}{Initial results - 1/2}
  \begin{minipage}[t]{0.45\textwidth}
  We implemented the algorithm on the following network, consisting different kinds of storage units, solar, gas and wind power for a time horizon of 5 weeks and time steps of one hour.
  \end{minipage}
  \begin{minipage}[t]{0.45\textwidth}
  \begin{figure}
    \includegraphics[scale = 0.5]{images/examplenetwork.png}
    \caption{Network layout}
  \end{figure}
  \end{minipage}
 

\end{frame}

\begin{frame}{Initial results - 2/2}
  In this instance the (not parallelized) algorithm converges to the optimal solutions in 12 iterations and in 0.46 seconds. Benders' algorithm converged in 0.44 seconds.
  \begin{figure}
    \includegraphics[scale=0.5]{images/ISPconvergence.png}
    \caption{Objective value of (ISP) for each iteration.}
  \end{figure}
  
\end{frame}

\begin{frame}{Conclusions.}
  \begin{itemize}
  \item The specific structure of the intertemporal constraints makes it possible do develop tailored optimization algorithms for (CEP). \pause
  \item The algorithm is analogous to a three stage bender decomposition. \pause (And I think the work by Filippo Pecci presented yesterday.)
  \end{itemize}
\end{frame}

\begin{frame}{Future Work.}
  \begin{itemize}
    \item We are currently implementing this and other stochastic methods within the Pypsa \cite{PyPSA} framework using the Linopy \cite{Hofmann2023} modeling package in Python. \pause
    \item Supporting hyperplanes for \(\hat{\cV_k(x, v_{t_k}, \omega)}\) could also be used for different \(k' \neq k\) and \(\omega' \neq \omega\), possibly decreasing the overall number of iterations to achieve convergence. \pause
    \item In general: equivalent LP formulations give different corresponding Hypergraph with different degrees of parallelizability.
    %\item Examine generalizations to multistage stochastic programs. \pause
  \end{itemize}

\end{frame}

\begin{frame}
  \vfill
  Thank you for your attention.

  \vfill
  \vfill
  \vfill
  \begin{footnotesize}
    gabor.riccardi01@universitadipavia.it \\
    \url{https://www.compopt.it}
  \end{footnotesize}
\end{frame}
\section{Bibliography}
\appendix
\begin{frame}{}
  Some references:\\[2em]
  \begin{footnotesize}
    \printbibliography[
      %heading=bibintoc,
      title={Bibliography}]
  \end{footnotesize}
\end{frame}

% \begin{frame}[noframenumbering]{Literature}
%   \begin{itemize}
%     \item Traditional stochastic capacity expansion methods, such as the L-shaped method, may perform poorly as the number of expansion possibilities increases.
%    
% \end{frame}



\begin{frame}[noframenumbering]{Power Grid Optimization}
  \begin{itemize}
    \item Optimal Power Flow (OPF) \hfill \textcolor{gray}{\cite{MathProgForm}}
          \begin{itemize}
            \item AC OPF: exact physical model
            \item Security-Constrained OPF (SCOPF) -- Includes contingencies to guarantee system security under failures.
            \item DC OPF and other linearized models \hfill \textcolor{gray}{\cite{LinRelBien} }
            \item other relaxations.
          \end{itemize}

    \item Unit Commitment -- Determines on/off status of power units, ignoring grid constraints.
    \item Economic Dispatch (ED) -- Minimizes generation cost, ignoring grid constraints.
  \end{itemize}
  % Blue arrow on the side of the text
  \begin{tikzpicture}[overlay, remember picture]
    \node at ([shift={(0.9,-1)}]current page.north west) (end0) {}; % Adjust for positioning
    \node at ([shift={(0.9,3)}]current page.south west) (start0) {}; % Adjust for positioning
    \draw[->, thick, blue] (start0) -- (end0) node[midway, fill=white, rotate=90] {Time / Exactness};

    \node at ([shift={(0.3,-1)}]current page.north west) (start1) {}; % Corrected position and syntax
    \node at ([shift={(0.3,3)}]current page.south west) (end1) {}; % Corrected position and syntax
    \draw[->, thick, yellow] (start1) -- (end1) node[midway, fill=white, rotate=90] {Stochasticity}; % Corrected spelling and syntax
  \end{tikzpicture}

  \vfill
  \textbf{Capacity expansion problem:} Based on Economic Dispatch models with added flow balance at bus nodes and various scenarios.
\end{frame}


%possible questions:
% 1) what is the relation between cep-a and the adequacy Assessment?
% Adequacy Assessment span multiple years, over which a certain capability of expanding the grid is expected.
% But the is also ma maximum expected capability to expand thegrid.
% 2) Comparison with L-shaped algorithm (i am a potato)


\end{document}








