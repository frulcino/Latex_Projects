%%%%%%%%%%%%%%%%%%%%%%%%%%%%%%%%%%%%%%%%%
% Long Lined Cover Letter
% LaTeX Template
% Version 2.0 (September 17, 2021)
%
% This template originates from:
% https://www.LaTeXTemplates.com
%
% Authors: Fanchao Chen
% (chenfc@fudan.edu.cn)
%
% License:
% CC BY-NC-SA 4.0 (https://creativecommons.org/licenses/by-nc-sa/4.0/)
%
%%%%%%%%%%%%%%%%%%%%%%%%%%%%%%%%%%%%%%%%%

%----------------------------------------------------------------------------------------
%	PACKAGES AND OTHER DOCUMENT CONFIGURATIONS
%----------------------------------------------------------------------------------------

\documentclass{article}

\usepackage{charter} % Use the Charter font

\usepackage[
	a4paper, % Paper size
	top=1in, % Top margin
	bottom=1in, % Bottom margin
	left=1in, % Left margin
	right=1in, % Right margin
	%showframe % Uncomment to show frames around the margins for debugging purposes
]{geometry}

\setlength{\parindent}{0pt} % Paragraph indentation
\setlength{\parskip}{1em} % Vertical space between paragraphs

\usepackage{graphicx} % Required for including images

\usepackage{fancyhdr} % Required for customizing headers and footers

\fancypagestyle{firstpage}{%
	\fancyhf{} % Clear default headers/footers
	\renewcommand{\headrulewidth}{0pt} % No header rule
	\renewcommand{\footrulewidth}{1pt} % Footer rule thickness
}

\fancypagestyle{subsequentpages}{%
	\fancyhf{} % Clear default headers/footers
	\renewcommand{\headrulewidth}{1pt} % Header rule thickness
	\renewcommand{\footrulewidth}{1pt} % Footer rule thickness
}

\AtBeginDocument{\thispagestyle{firstpage}} % Use the first page headers/footers style on the first page
\pagestyle{subsequentpages} % Use the subsequent pages headers/footers style on subsequent pages

%----------------------------------------------------------------------------------------

\begin{document}

%----------------------------------------------------------------------------------------
%	FIRST PAGE HEADER
%----------------------------------------------------------------------------------------

%\includegraphics[height=2.4cm]{siopt/unipv-logo.pdf} \hspace{2.4cm} 
%\includegraphics[height=2.4cm]{siopt/usi-logo.pdf}% Logo

\vspace{-1em} % Pull the rule closer to the logo

\rule{\linewidth}{1pt} % Horizontal rule

\bigskip\bigskip % Vertical whitespace

%----------------------------------------------------------------------------------------
%	YOUR NAME AND CONTACT INFORMATION
%----------------------------------------------------------------------------------------

\hfill
\begin{tabular}{l @{}}
	\today \bigskip\\ % Date
	Ambrogio Maria Bernardelli \\
	via Ferrata 5, 27100 Pavia, Italy \\ % Address
	%Address 2 \\
	%Phone: (000) 111-1111 \\
	Email: ambrogiomaria.bernardelli01@universitadipavia.it
\end{tabular}

\bigskip % Vertical whitespace

%----------------------------------------------------------------------------------------
%	ADDRESSEE AND GREETING
%----------------------------------------------------------------------------------------

\textbf{Subject:} \textit{On the integrality gap of the Complete Metric Steiner Tree Problem via a novel formulation}\\
$\qquad$ A.M. Bernardelli, E. Vercesi, S. Gualandi, M. Mastrolilli, and L. M. Gambardella

\bigskip % Vertical whitespace

Dear Professor Jong-Shi Pang, 

\bigskip % Vertical whitespace

Please find attached our manuscript titled ``On the integrality gap of the Complete Metric Steiner Tree Problem via a novel formulation'' for consideration in SIAM Journal on Optimization. 
%
Part of the work of this paper has already been presented as an extended abstract at the International Symposium on Combinatorial Optimization (ISCO) 2024 and as a poster in an informal poster session at 
Integer Programming and Combinatorial Optimization (IPCO) 2024.

In this paper, we aim to extend Benoit, Boyd, and Elliott-Magwood’s foundational work on the integrality gap in the Metric Symmetric/Asymmetric Traveling Salesman Problem to the Metric Steiner Tree Problem. 
%

We introduce the Complete Metric formulation, specifically designed to address the weakness of the well-established bidirected cut formulation on metric instances.

Our formulation refine the polytope structure of the Metric Steiner Tree Problem, resulting in a polytope $ P_{CM} $ with a significantly reduced number of vertices compared to the existing directed cut formulation.

We introduce two new heuristic algorithms for enumerating vertices of $P_{CM}$, that facilitate the computation of vertices with the largest known integrality gap for instances up to 10 vertices.


Our computational results reveal notable instances, including small-node examples with high integrality gap, which realize the best-known lower bounds on the integrality gap for the directed cut formulation.


\bigskip % Vertical whitespace

Thank you for considering our manuscript, we look forward to receiving your feedback.

Best Regards,

\vspace{\fill} % Vertical whitespace
%\includegraphics[height=1.6cm]{siopt/firma.png}\\ 
Ambrogio Maria Bernardelli\\
PhD Student\\
University of Pavia -- Università della Svizzera Italiana

\end{document}