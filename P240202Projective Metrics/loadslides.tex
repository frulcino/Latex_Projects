
% Copyright (c) 2022 by Lars Spreng
% This work is licensed under the Creative Commons Attribution 4.0 International License. 
% To view a copy of this license, visit http://creativecommons.org/licenses/by/4.0/ or send a letter to Creative Commons, PO Box 1866, Mountain View, CA 94042, USA.

%~~~~~~~~~~~~~~~~~~~~~~~~~~~~~~~~~~~~~~~~~~~~~~~~~~~~~~~~~~~~~~~~~~~~~~~~~~~~~~
% Add your packages and commands to this file
%~~~~~~~~~~~~~~~~~~~~~~~~~~~~~~~~~~~~~~~~~~~~~~~~~~~~~~~~~~~~~~~~~~~~~~~~~~~~~~
% MY PACKAGES

\usepackage{tikz}
\usepackage[outline]{contour}

\usetikzlibrary {arrows.meta,decorations.text, decorations.pathmorphing, decorations.pathreplacing, decorations.shapes,
}

\usepackage{amsmath}
\usepackage{amsfonts}
\usepackage{amssymb}
\usepackage{eurosym}
\usepackage{mathtools}
\usepackage{float}
\usepackage{xfrac}
\usepackage{mathrsfs} 
%~~~~~~~~~~~~~~~~~~~~~~~~~~~~~~~~~~~~~~~~~~~~~~~~~~~~~~~~~~~~~~~~~~~~~~~~~~~~~~
\RequirePackage{palatino}
\RequirePackage[utf8]{inputenc}
\RequirePackage[T1]{fontenc}
\RequirePackage{}
\usepackage{intcalc}
\usefonttheme{serif}

\usepackage{styles/elegantmacros}
\usefolder{styles}
\usetheme[style=blue]{elegant}



\newcommand{\makepart}[1]{ % For convenience
\part{#1} \frame{\partpage}
}

%~~~~~~~~~~~~~~~~~~~~~~~~~~~~~~~~~~~~~~~~~~~~~~~~~~~~~~~~~~~~~~~~~~~~~~~~~~~~~~

%~~~~~~~~~~~~~~~~~~~~~~~~~~~~~~~~~~~~~~~~~~~~~~~~~~~~~~~~~~~~~~~~~~~~~~~~~~~~~~
% Figures
\RequirePackage{booktabs}
\RequirePackage{colortbl}
\RequirePackage{ragged2e}
\RequirePackage{schemabloc}
%\RequirePackage{natbib}
\RequirePackage{caption}
\RequirePackage{subcaption}
\RequirePackage{tabularx}
\RequirePackage{colortbl}
\RequirePackage{array}
\RequirePackage{multirow}
\usepackage[
  style=authoryear, 
]{biblatex}
\addbibresource{references.bib}
\newcolumntype{Y}{>{\centering\arraybackslash}X}

%~~~~~~~~~~~~~~~~~~~~~~~~~~~~~~~~~~~~~~~~~~~~~~~~~~~~~~~~~~~~~~~~~~~~~~~~~~~~~~

%~~~~~~~~~~~~~~~~~~~~~~~~~~~~~~~~~~~~~~~~~~~~~~~~~~~~~~~~~~~~~~~~~~~~~~~~~~~~~~
% Figures
\RequirePackage{wrapfig}
\RequirePackage{pgfplots}
\pgfplotsset{compat=1.18}
\RequirePackage{graphicx}
\RequirePackage{adjustbox}
\RequirePackage{environ}
%\pgfplotsset{compat=1.18}


\makeatletter
\newsavebox{\measure@tikzpicture}
\NewEnviron{scaletikzpicturetowidth}[1]{%
  \def\tikz@width{#1}%
  \def\tikzscale{1}\begin{lrbox}{\measure@tikzpicture}%
  \BODY
  \end{lrbox}%
  \pgfmathparse{#1/\wd\measure@tikzpicture}%
  \edef\tikzscale{\pgfmathresult}%
  \BODY
}
\makeatother
%~~~~~~~~~~~~~~~~~~~~~~~~~~~~~~~~~~~~~~~~~~~~~~~~~~~~~~~~~~~~~~~~~~~~~~~~~~~~~~

%~~~~~~~~~~~~~~~~~~~~~~~~~~~~~~~~~~~~~~~~~~~~~~~~~~~~~~~~~~~~~~~~~~~~~~~~~~~~~~
% Maths 
\RequirePackage{textcomp}
\RequirePackage{amsmath} 
\RequirePackage{amsthm}
\RequirePackage{mathtools}
%\RequirePackage{bbm}
%\RequirePackage{algorithm}
%\RequirePackage[osf,sc]{mathpazo}
%\RequirePackage{pifont}
%\newcommand{\xmark}{\ding{55}}%
%\numberwithin{equation}{section}
\DeclareMathOperator*{\argmax}{arg\,max}
\DeclareMathOperator*{\argmin}{arg\,min}

% Define mathematical symbols


\newcommand{\nc}{\newcommand}
\newcommand{\cN}{\mathcal{N}}
\newcommand{\cG}{\mathcal{G}}
\newcommand{\cE}{\mathcal{E}}
\newcommand{\cW}{\mathcal{W}}
\newcommand{\cD}{\mathcal{D}}
\nc{\cV}{\mathcal{V}}
\nc{\cT}{\mathcal{T}}
\nc{\bE}{\mathbb{E}}


%parameters and variables
\nc{\PV}{\mathcal{PV}}
\nc{\HL}{\mathcal{H}^L}
\nc{\PL}{\mathcal{P}^L}
\nc{\WW}{\mathcal{W}}
\nc{\ns}{\text{ns}}
\nc{\nw}{\text{nw}}
\nc{\nh}{\text{nh}}
\nc{\PP}{\text{P}}
\nc{\HH}{\text{H}}
\nc{\HS}{\text{H}^S}
\nc{\CG}{C_{\Sigma}^{\text{Gauss}}}
\usetikzlibrary{shapes,arrows}
\usetikzlibrary{overlay-beamer-styles}

\tikzstyle{block} = [draw, fill=blue!20, rectangle, minimum height=4em, minimum width=6em]
\tikzstyle{circleblock} = [draw, fill=red!20, circle, minimum height=4em]
\tikzstyle{line} = [draw, -latex]


\nc{\bfone}{\mathbf{1}}

\nc{\ppf}{\mathsf{ppf}}
\nc{\PR}{\mathbb{P}}
\newcommand{\spn}[1]{\left\langle #1 \right\rangle}
\newcommand{\supp}{\operatorname{Supp}}

\newcommand{\withoutzero}{\setminus\{0\}}
%\newcommand{\wt}{\operatorname{wt}}
\newcommand{\stab}{\operatorname{Stab}}
%\renewcommand{\Span}{\operatorname{Span}}
%\newcommand{\isom}{\operatorname{Isom}}
%\renewcommand{\isom}%{\operatorname{Isom}}
\renewcommand{\ker}{\operatorname{Ker}}
\setbeamertemplate{theorems}[numbered] % to number

\theoremstyle{definition}
\newtheorem{fact}{Fact}[section]
\newtheorem{examp}{Example}[section]

\theoremstyle{plain}
\newtheorem{definition}{Definition}[section]
\newtheorem{proposition}{Proposition}
\newtheorem{theorem}{Theorem}
\newtheorem{assumption}{Assumption}

%%cell command
\newcommand{\cell}[2]{%
    \begin{tabular}{|c|}
     \hline \cellcolor{#2} #1 \\ \hline
    \end{tabular}
}

\providecommand{\H}{\mathscr{H}}      
\providecommand{\E}{\mathbb{E}}
\makeatletter
\def\munderbar#1{\underline{\sbox\tw@{$#1$}\dp\tw@\z@\box\tw@}}
\makeatother

%~~~~~~~~~~~~~~~~~~~~~~~~~~~~~~~~~~~~~~~~~~~~~~~~~~~~~~~~~~~~~~~~~~~~~~~~~~~~~~

%\usepackage{xcolor}

