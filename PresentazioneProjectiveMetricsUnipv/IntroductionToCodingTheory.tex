\begin{frame}{Introduction to Coding Theory}

    \begin{itemize}
        \item Retrieve information from corrupted messages \pause
        \item Store data (which is just sending a message to your future self) \pause
        \item Data compression \pause
        \item Cryptography 
    \end{itemize}
\end{frame}
\begin{frame}

    \vspace{-2cm} % Adjust the vertical space as needed
    \hspace{2.3cm}\textcolor{red}{How?} \pause 
    \begin{center}
        \alert<1>{\Huge Redundancy!} \\ \pause
        \alert<3>{\huge (Smartly)}
    \end{center}
    \pause
    Let's repeat each row of a message three times and then send it.\\ \pause
    If we obtain the message:
    \pause
    \only<5>{
    \begin{center}
         \begin{tabular}{|c|c|c|c|c|c|c|c|c|}
        \hline
        0 & 0 & 0 & 1 & 0 & 1 & 0 & 0 & 0 \\
        \hline
        0 & 0 & 0 & 1 & 1 & 1 & 0 & 0 & 0 \\
        \hline
        0 & 0 & 0 & 1 & 1 & 1 & 0 & 0 & 0 \\
        \hline
        \end{tabular}
    \end{center}
    }
    
    \uncover<6->{
    \begin{center}
         \begin{tabular}{|c|c|c|c|c|c|c|c|c|}
        \hline
        0 & 0 & 0 & 1 & \cellcolor{red!20}0 & 1 & 0 & 0 & 0 \\
        \hline
        0 & 0 & 0 & 1 & \cellcolor{red!20} 1 & 1 & 0 & 0 & 0 \\
        \hline
        0 & 0 & 0 & 1 & \cellcolor{red!20} 1 & 1 & 0 & 0 & 0 \\
        \hline
        \end{tabular}
            
    \end{center}
    }
    \pause
    \bigskip
    
    Since "1" appears twice and "0" once, we may assume more likely that the original message was:
    \pause
    \bigskip
    
    \begin{center}
    \begin{tabular}{|c|c|c|c|c|c|c|c|c|}
        \hline
        0 & 0 & 0 & 1 & \cellcolor{green!20} 1 & 1 & 0 & 0 & 0 \\
        \hline
    \end{tabular}
    \end{center}
    \pause
    \bigskip
    
    But this way, we took three times the length of the message to correct one error!
    
    \bigskip
    \pause
    \textcolor{gray}{Btw the message says SOS in Morse, so maybe go seek help.}
    \end{frame}
    
\begin{frame}{A better way to do it}
We rearrange the original message
\begin{tabular}{|c|c|c|c|c|c|c|c|c|}
    \hline
    \rowcolor{blue!20} 0 & 0 & 0 & 1 &  1 & 1 & 0 & 0 & 0 \\
    \hline
\end{tabular}
in a 4 by 4 grid: \\ 
\begin{center}
\pause
\begin{tabular}{|c|c|c|c|}
\hline
     &  \cellcolor{red!20} 0 & \cellcolor{red!20} 0 & \cellcolor{blue!20}  0  \\ \hline
    \cellcolor{red!20} 1 & \cellcolor{blue!20} 0 & \cellcolor{blue!20}  0 & \cellcolor{blue!20} 1  \\ \hline
    \cellcolor{red!20} 0 & \cellcolor{blue!20} 1 &\cellcolor{blue!20}  1 &\cellcolor{blue!20} 0  \\ \hline
    \rowcolor{blue!20}0 & 0 & 0 & 0  \\ \hline
\end{tabular}
\end{center}
\pause
Where the red cells are the four redundant bits which encode the parity of subsets of columns or rows in the following way. \\
\pause
\only<4>{   
\begin{minipage}{0.4\textwidth}
        \begin{center}
    \begin{tabular}{|c|c|c|c|}
        \hline
          &  \cellcolor{red!35} 0 & 0 & \cellcolor{blue!20}  0  \\ \hline
        1 & \cellcolor{blue!20} 0 &  0 & \cellcolor{blue!40} 1  \\ \hline
        0 & \cellcolor{blue!40} 1 &  1 &\cellcolor{blue!20} 0  \\ \hline
        0 & \cellcolor{blue!20} 0 & 0 &  \cellcolor{blue!20} 0  \\ \hline
\end{tabular}
\end{center}
\end{minipage}
%
\begin{minipage}{0.4 \textwidth}
    \begin{center}
        \cell{0}{red!35} = \cell{1}{blue!40} + \cell{1}{blue!40}
    \end{center}
\end{minipage}
}
\only<5>{   
\begin{minipage}{0.4\textwidth}
        \begin{center}
    \begin{tabular}{|c|c|>{\columncolor{blue!20}}c|>{\columncolor{blue!20}}c|c|} 
    \hline
          &  0 & \cellcolor{red!30}0  &  0  \\ \hline
        1 & 0 &  0 & \cellcolor{blue!40}1  \\ \hline
        0 & 1 &  \cellcolor{blue!40}1 & 0  \\ \hline
        0 & 0 & 0 &  0  \\ \hline
\end{tabular}
\end{center}
\end{minipage}
%
\begin{minipage}{0.4 \textwidth}
    \begin{center}
        \cell{0}{red!35} = \cell{1}{blue!40} + \cell{1}{blue!40}
    \end{center}
\end{minipage}
}
\only<6>{   
\begin{minipage}{0.4\textwidth}
        \begin{center}
    \begin{tabular}{|c|c|c|c|}
        \hline
          &  0 & 0 &  0  \\ \hline
        \rowcolor{blue!20} \cellcolor{red!30}1 & 0 &  0 & \cellcolor{blue!40}1  \\ \hline
        0 & 1 &  1 & 0  \\ \hline
        \rowcolor{blue!20} 0 & 0 & 0 &  0  \\ \hline
\end{tabular}
\end{center}
\end{minipage}
%
\begin{minipage}{0.4 \textwidth}
    \begin{center}
        \cell{1}{red!35} = \cell{1}{blue!40} 
    \end{center}
\end{minipage}
}
\uncover<7->{
\begin{minipage}{0.4\textwidth}
        \begin{center}
    \begin{tabular}{|c|c|c|c|}
        \hline
          &  0 & 0 &  0  \\ \hline
        1 & 0 &  0 & 1  \\ \hline
        \rowcolor{blue!20} \cellcolor{red!35}0 & \cellcolor{blue!40} 1 &  \cellcolor{blue!40} 1 & 0  \\ \hline
        \rowcolor{blue!20} 0 & 0 & 0 &  0  \\ \hline
\end{tabular}
\end{center}
\end{minipage}
%
\begin{minipage}{0.4 \textwidth}
    \begin{center}
        \cell{0}{red!35} = \cell{1}{blue!40} + \cell{1}{blue!40}
    \end{center}
\end{minipage}
}
\hfill \\
\bigskip
\pause
\uncover<8->{
Since \(2^4 = \text{total number of bits} + (\text{case in which there is no error}) = 15 + 1\) and if there is up to one error, every redundant bit halvens the number the possible locations of where the error might be, we can always correct up to one error in the message.}
\end{frame}
\begin{frame}{Hamming Codes}
    
    The subset of codes in $\bF_2^{15}$ constructed the same way are called \textbf{Hamming Codes}.  \pause \\
    \vspace{0.5cm}
    \begin{minipage}{0.25\textwidth}
        \centering
        \begin{tabular}{|c|c|c|c|}
            \hline
            &  \cellcolor{red!20} 0 & \cellcolor{red!20} 0 & \cellcolor{blue!20}  1 \\ \hline
            \cellcolor{red!20} 0 & \cellcolor{blue!20} 0 & \cellcolor{blue!20}  0 & \cellcolor{blue!20} 1  \\ \hline
            \cellcolor{red!20} 1 & \cellcolor{blue!20} 0 &\cellcolor{blue!20}  1 &\cellcolor{blue!20} 1  \\ \hline
             \rowcolor{blue!20}0 & 1 & 0 & 0  \\ \hline
        \end{tabular}
    \end{minipage}
    %
    \begin{minipage}{0.05\textwidth}
    \centering
        \textbf{+}
    \end{minipage}
    %
    \begin{minipage}{0.25\textwidth}
    \centering
        \begin{tabular}{|c|c|c|c|}
            \hline
            &  \cellcolor{red!20} 0 & \cellcolor{red!20} 0 & \cellcolor{blue!20}  0 \\ \hline
            \cellcolor{red!20} 1 & \cellcolor{blue!20} 0 & \cellcolor{blue!20}  0 & \cellcolor{blue!20} 1  \\ \hline
            \cellcolor{red!20} 0 & \cellcolor{blue!20} 1 &\cellcolor{blue!20}  1 &\cellcolor{blue!20} 0  \\ \hline
             \rowcolor{blue!20}0 & 0 & 0 & 0  \\ \hline
        \end{tabular}
    \end{minipage}
    %
    \begin{minipage}{0.05\textwidth}
    \centering
        \textbf{=}
    \end{minipage}
%
    \begin{minipage}{0.25\textwidth}
    \centering
        \begin{tabular}{|c|c|c|c|}
            \hline
            &  \cellcolor{red!20} 0 & \cellcolor{red!20} 0 & \cellcolor{blue!20}  1 \\ \hline
            \cellcolor{red!20} 1 & \cellcolor{blue!20} 0 & \cellcolor{blue!20}  0 & \cellcolor{blue!20} 0  \\ \hline
            \cellcolor{red!20} 1 & \cellcolor{blue!20} 1 &\cellcolor{blue!20}  0 &\cellcolor{blue!20} 1  \\ \hline
             \rowcolor{blue!20}0 & 1 & 0 & 0  \\ \hline
        \end{tabular}
    \end{minipage} \\
\bigskip \pause
We observe that if we sum two Hamming codes, it remains an Hamming code (that is the parity checks remain valid also for the result of the sum): \\
\pause
     Thus, since we can choose the numbers inside the 11 blue cells arbitrarily they form a 11 dimensional linear subspace of $\bF_2^{15}$.  For this reason these codes are referred to as \textcolor{blue}{[15,11] Hamming Codes}.
     
\end{frame}

\begin{frame}{What are we actually doing when decoding a message?}
\begin{itemize}
    \item Note given a code in \(\bF^{15}_2\) by changing one bit we can always recover an Hamming Code. Thus what we are doing to correct a message is simply taking the closest valid message! 
    \pause
    \item We partition the message space into balls centered on the codewords. If we receive a message, then we simply look at what ball it is in and then the center of that ball is the most likely correct message (This is called \textcolor{blue}{nearest neighbour decoder (nnd)}).
    \pause
    \item To decode as many messages as possible, we take the largest radius such that the balls remain disjoint. 
    \pause
    \item This radius equals to \( \lfloor \frac{d-1}{2}\rfloor\) where \(d\) is \emph{the minimum distance} of the code (the set containing all the \(m\) codewords).
    \pause
    \item The minimum distance \(d\) is a simple measure of the goodness of a code.
    %\item (If the bit errors are i.i.d. then the nnd equals the maximum likelyhood decoder.)
    
    
   

    %if we center around each codeword in the alphabet a ball of size r and the balls are disjoint then r represents the number of errors we can correct in a code! We want the biggest r given the size of an alphabet-----) Sphere Packing
    
\end{itemize}

\end{frame}

\begin{frame}{Classical Sphere Packing Results}
\begin{itemize}
    \item Given a message length \(n\) and a minimum distance \(d\), we want to find the largest code with minimum distance \(d\). \\
    \item This can also be seen as the sphere packing problem for spheres of radius \(\lfloor \frac{d-1}{2}\rfloor\). 
\end{itemize}
\quad

\begin{theorem}[Hamming Bound] \label{hamming}
    Let \(\cC < \bF_q^n \) be a code with \(d_H(\cC) = d\) then:
\[|\cC|\leq \frac{q^n}{\sum_{i=o}^t\binom{n}{i}(q-1)^i}\] \\

Where \(t \coloneqq \lfloor \frac{d-1}{2}\rfloor \).
\end{theorem}

\begin{theorem}[Singleton Bound] \label{singleton}
Let \(\cC < \bF_q^n \) be a code with \(d_H(\cC) = t\) then: \[ |\cC| \leq q^{n-d+1}.\] 
\end{theorem}
\vspace{0.5cm}
The codes satisfying the Hamming bound or the Singleton bound are called respectively \textbf{Perfect codes} and \textbf{MDS codes} (maximum distance separable codes).
%\begin{frame}{In general}

% \begin{tikzpicture}[scale = 0.5]
%   % Draw a grid of points
%   \foreach \x in {0,...,16}
%     \foreach \y in {0,...,16}
%       \fill[black] (\x, \y) circle (2pt);

%     \foreach \x in {0,...,5}
%         \foreach \y in {0,...,5}
%             \fill[blue] (\x*3, \y*3) circle (3pt); 
              

%   % Color some points differently
%   \fill[red] (1,2) circle (2pt);
%   \fill[blue] (3,1) circle (2pt);

%   % Draw arrows between points
%   \draw[->, thick] (0,0) -- (1,2);
%   \draw[->, thick] (2,3) -- (3,1);
% \end{tikzpicture}
% %
\end{frame}
%drawing idea: to balls with half the radious of distance
\begin{frame}{Proof of Hamming bound}
Let \(\mathcal{C} \subset \mathbb{F}_q^n\) be a code. \pause \(t = \lfloor \frac{d-1}{2}\rfloor\) is the maximum radius such that the union \(\cup_{c \in \cC}B_t(c)\) is disjoint. 
\pause Then \(|\sqcup_{c \in \cC}B_t(c)|\leq |\bF_q^n|=q^n\) and \pause \(|\cup_{c \in \cC}B_t(c)| = |\cC|b_t\). Where \(b_t\) is the size of the ball of radius t. \pause Since \(b_t = \sum_{i=0}^ts_i\) where \(s_i\) is the size of the sphere of radius \(i\). \\ \pause The latter equals the number of way we can choose exactly \(i\) non null coordinates in a vector of length \(n\), thus \(s_i = \binom{n}{i}(q-1)^i\).\pause Thus \(|\cup_{c \in \cC}B_t(c)| = |\cC|\sum_{i=0}^t \binom{n}{i}(q-1)^i \leq q^n \). 

% Let \(c,c' \in \cC\) be two distinct codewords with distance \(d\) and \(x \in B_t(c),x' \in B_t(x')\).
% Then \(d = d(c,c') \leq d(c,x) + d(x,x') + d(x',c') \leq t + 1 + t\). Thus \( \lfloor \frac{d-1}{2}\rfloor \leq t\). \(c,c'\) have exactly \(d\) different coordinates having indices \(i_1,\ldots,i_d\). Let \(x\) be the vector having coordinates \(x_j = c_j\) when \(j \neq i_1,\ldots,i_{\lfloor \frac{d-1}{2}\rfloor}\) and \(x_j = c'_j \) otherwise. Then \( d(x,c') =  \lfloor \frac{d-1}{2} \rfloor \leq t\) thus \(x \in B_t(c')\) and \(x \notin B_t(c)\). Thus \( t < d(x,c) = d - \lfloor \frac{d-1}{2}\rfloor = d + \lceil \frac{1-d}{2}\rceil = \lceil d +\frac{1-d}{2}\rceil= \lceil \frac{d+1}{2}\rceil\) thus \(t \leq \lfloor \frac{d-1}{2}\rfloor\).

\vspace{1cm}
We observe how knowing the sphere size was a central part of the proof. \\ \pause
\vspace{1cm}
\textbf{Remark:} 1) Codes are perfect if the balls of size t centered on the codewords completely fill up \(\cV\) \\2)The Hamming codes are Perfect codes, while the "send the same message multiple times"-codes are not perfect.

\end{frame}



\begin{frame}{Other Metrics}{}
	
Hamming metric
\begin{equation*}
\left(\begin{array}{c>{\columncolor{red!20}}ccc>{\columncolor{red!20}}cc>{\columncolor{red!20}}c}
0  & 1  & 0 & 0 & 1 & 0 & 1\\
\end{array}\right) \quad \to \;\wt_{\operatorname{Hamming}} = 3
\end{equation*}

\uncover<2->{
Rank metric

\begin{equation*}
\left(\begin{array}{>{\columncolor{blue!20}}c>{\columncolor{blue!20}}c>{\columncolor{blue!20}}c}
0  & 1  & 0 \\
1  & 1  & 0 \\
1  & 1  & 1 \\
\end{array}\right)  \quad \to \;\wt_{\operatorname{Rank}} = 3
\end{equation*}
}


\uncover<3->{
Cover metric (rows and columns)

\begin{equation*}
  \left(\begin{array}{c>{\columncolor{red!20}}cccc}
    0  & 1  & 0 & 0 & 0\\
    \rowcolor{red!20}
    0  & 1  & 0 & 1 & 1\\
    0  & 1  & 0 & 0 & 0\\
    \rowcolor{red!20}
    1  & 0  & 0 & 1  & 0\\
  \end{array}\right)  \quad \to \;\wt_{\operatorname{Cover}} = 3
\end{equation*}
}

\uncover<4->{
Phase-rotation metric

\only<1-4>{
\begin{equation*}
\left(\begin{array}{cccc}
1  & 1  & 0 & 1\\
\end{array}\right) = \left(\begin{array}{cccc}
\rowcolor{red!20}
1  & 1  & 1 & 1\\
\end{array}\right) + \left(\begin{array}{cc>{\columncolor{red!20}}cc}
0  & 0 & 1 &  0\\
\end{array}\right)  \quad \to \;\wt_{\operatorname{Phase-Rot}} = 1+1 = 2
\end{equation*}
}
\only<5-5>{
\text{}\\ More: burst metric, tensor metric, combinatorical metrics, etc.\\ \text{}
}
}
\end{frame}




% \begin{frame}{Other metrics}
%    \begin{itemize}
%    \item The Hamming metric is suitable when errors occur bit-wise with equal probability. For different error scenarios, alternative metrics must be considered. 
%    \pause
%    \item
%    For each metric it is important to try to get analogous results to \ref{hamming}  and \ref{singleton} bounds. 
%    \pause
%    \item In particular we are interested in the sphere sizes
%    \pause
%    \item
%    Generally each metric is studied individually.
%    \pause
%    \item
%    A different approach is to try to get analogous results on a family of metrics.
%    \pause
%    \item In this presentation we show our results on the family of projective metrics.
%    \end{itemize}
% \end{frame}