\documentclass[12pt,a4paper,oneside,openany]{book}
%Preambolo
\usepackage[utf8]{inputenc}
\usepackage[italian]{babel,varioref}
\usepackage[T1]{fontenc}
\usepackage[suftesi]{frontespizio} 
\usepackage{amsthm}
\usepackage{amsmath}
\usepackage{amsfonts}
\usepackage{amssymb}
\usepackage{graphicx}
\usepackage{rotating}
\usepackage{setspace}
\usepackage{color}
\usepackage{cite}
\usepackage{fancyhdr}
\usepackage{ragged2e}
\usepackage{appendix}
\usepackage{tabularx}
\usepackage{multirow}
\usepackage{booktabs}
\usepackage{pgfplots}
\usepackage{pgfmath}
\usepackage{url}
\usepackage{emptypage}

%questo mi dà la quadrettatura
%\usepackage[showframe]{geometry}
%\savegeometry{origin}
%\geometry{rmargin=2.6cm,lmargin=2.6cm,tmargin=3cm,bmargin=2.7cm}% for the title page
\usepackage[utf8]{inputenc}

% Default fixed font does not support bold face
\DeclareFixedFont{\ttb}{T1}{txtt}{bx}{n}{7} % for bold
\DeclareFixedFont{\ttm}{T1}{txtt}{m}{n}{7}  % for normal

% Custom colors
\usepackage{color}
\definecolor{deepblue}{rgb}{0,0,0.5}
\definecolor{deepred}{rgb}{0.6,0,0}
\definecolor{deepgreen}{rgb}{0,0.5,0}

\usepackage{listings}

% Python style for highlighting
\newcommand\pythonstyle{\lstset{
language=Python,
numbers=left,
numberstyle=\tiny\color{black},
basicstyle=\ttm,
otherkeywords={self},             % Add keywords here
keywordstyle=\ttb\color{deepblue},
emph={MyClass,__init__},          % Custom highlighting
emphstyle=\ttb\color{deepred},    % Custom highlighting style
stringstyle=\color{deepgreen},
frame=tb,                         % Any extra options here
showstringspaces=false            % 
}}


% Python environment
\lstnewenvironment{python}[1][]
{
\pythonstyle
\lstset{#1}
}
{}

% Python for external files
\newcommand\pythonexternal[2][]{{
\pythonstyle
\lstinputlisting[#1]{#2}}}

% Python for inline
\newcommand\pythoninline[1]{{\pythonstyle\lstinline!#1!}}


\usepackage{collectbox}

\newcommand{\mybox}[2]{$\quad$\fbox{
\begin{minipage}{#1cm}
\hfill\vspace{#2cm}
\end{minipage}
}}


\newtheorem{esempio}{Esempio}
\newcommand{\virgolette}[1]{``#1''} %\virgolette{questo č fra virgolette}%
%\usepackage{frontespizio}
\newcommand{\abs}[1]{\lvert#1\rvert}


%%%%%%%%%%%%%%%%%%%%%%%%%%%%%%%%% PARTE PRINCIPALE %%%%%%%%%%%%%%%%%%%%%%%%%%%%%%%%%%%%
\begin{document}

\frontmatter
\thispagestyle{empty}
\begin{center}
	\hrulefill \\
\end{center}

\begin{center}
	\includegraphics[width=3cm]{unipv.png}\\
	
	\sc{\bf UNIVERSITÀ DEGLI STUDI DI PAVIA} \\
	{\bf DIPARTIMENTO DI MATEMATICA \virgolette{FELICE CASORATI}}\\
	%{\footnotesize Direttore Ch.ma Prof.ssa Antonella Profumo}
	\vspace*{1cm}
	
	
%	 \vspace{8pt} \\
	%\textbf{--------------------------}
	\vspace{-8pt}
	\normalsize Corso di Laurea Triennale in Matematica
\end{center}
\vspace{2cm}


\begin{center}
	\large 	TITOLO CHE VOGLIO METTERE ALLA MIA TESI, ANCHE SE E' LUNGO NON E' UN PROBLEMA
\end{center}

\vspace*{3cm}

\begin{flushleft}
	\begin{tabular}{ll}
		{\sc Relatore:} & {\sc nome cognome} \vspace{5pt} \\
		{\sc Correlatrice:} & {\sc nome cognome} \\
		%& Ch.mo Prof. {\sc ???}
	\end{tabular}
\end{flushleft}

\vfill

\begin{flushright}
	\textsc{Laureando:} \\
	{\sc Nome Cognome} \\
	Matricola: $00000$
\end{flushright}

\vfill

\begin{center}
	\hrulefill \\
	\small Anno Accademico $2020-2021$ \\ 
\end{center}

%------dedica---------------------
\cleardoublepage
\thispagestyle{empty}
\vspace*{\stretch{1}}
\begin{flushright}
	%\itshape Dedica
	\itshape dedica\\
	che voglio fare
\end{flushright}
\vspace{\stretch{2}}
\cleardoublepage

\mainmatter

\begin{justify}
%da togliere il commento se le voglio
%\tableofcontents
%\listoffigures
%\listoftables

\chapter*{Introduzione}
L'obiettivo di questa tesi è ...

SCRIVERE BREVE RIASSUNTO DELLA TESI 1-2 PAGINE AL MASSIMO


\chapter{Titolo del capitolo}\label{chap:1}
Introduzione al problema o alla tematica trattata nella tesi.
I riferimenti principali sono il libro \cite{dantzig2016linear} e l'articolo \cite{dantzig1960decomposition}.


\section{Sezione del capitolo}\label{sec:nomeDaRicordare}

\subsection{Sotto sezione del capitolo}\label{sec:nomeDaRicordare}

Esempio di lista numerata:
\begin{enumerate}
\item primo caso
\item secondo caso
\item ...
\item ultimo caso
\end{enumerate}

Esempio di lista non numerata:
\begin{itemize}
\item primo caso
\item secondo caso
\item ...
\item ultimo caso
\end{itemize}

\paragraph{Sottoparagrafo.} Ecco il testo del paragrafo...
\chapter[Titolo corso]{Titolo lungo, lungo, lungo che non entra in alto nella testi}\label{chap:lungo}
In questo capitol vediamo i modelli.

\section{Modelli}

ESEMPIO DI MODELLO:
\begin{align}
\label{mod1:fobj} \min \quad & \sum_{i \in I}\sum_{j \in J} c_{ij} x_{ij} \\
\label{mod1:vinc1}           & \sum_{j \in J} a_{ij} x_{ij} \geq b_i & \forall i \in I \\
\label{mod1:vinc2}           & \sum_{i \in I} a_{ij} x_{ij} \leq c_j & \forall j \in J \\
\label{mod1:varx}            &_{ij} x \in \{0,1\} & \forall i \in I, \forall j \in J. \\
\end{align}

La \eqref{mod1:fobj} rappresenta la funzione obiettivo ...
La \eqref{mod1:vinc1} esprimi i vincoli di bla bla bla.
La \eqref{mod1:vinc2} impone i vincol per bla bla bla.
Le variabili binari \eqref{mod1:varx} assumono i valori quando ..

\begin{esempio}[Esempio di equazione]\label{esempio:a}
Si possono anche scrivere esempio. Per esempio sintassi per le equazioni numerate è la seguente:
\begin{equation}\label{eq:einstein}
e = m c^2
\end{equation}
\end{esempio}
\chapter[Risultati computazionali]{Risultati computazionali}
Esempio di uso di tabelle:

\begin{table}[ht!]
\begin{center}
\begin{tabular}{l|cr}
uno & due & tre \\
\hline
0 & 2323 & 1212 \\
1 & 2909& 099.2
\end{tabular}
\end{center}
\caption{Esempio minimale di tabella.}\label{tab:2}
\end{table}


\paragraph{Dettagli implementativi.} I test sono stati eseguiti su un portatile ...


\chapter{Conclusioni}\label{chap:end}

Riassunto conclusivo dell tesi.

\appendix
\chapter{Acronimi}
\begin{itemize}
\item TSP: Travelling Salesman Problem 
\item VRP: Vehicle Routing Problem 
\item BPP: Bin Packing Problem
\item ...
\end{itemize}





\chapter{Notazione}
\begin{itemize}
	\item miao
\end{itemize}












\end{justify}


% -------- BIBLIOGRAFIA --------
\bibliographystyle{plain}
\bibliography{bibliografia}

\end{document}
