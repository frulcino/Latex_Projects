\documentclass{article}

\usepackage{DejaVuSans}
\usepackage[T1]{fontenc}
\usepackage{tabularray}
\usepackage{csvsimple-l3}
\usepackage{tikz}
\usepackage{tcolorbox}
\usepackage{siunitx}
\usepackage{utfsym}
\usepackage[a4paper, landscape, margin=1cm]{geometry}
\usepackage{xfp} % Correct package for \fpeval

\renewcommand\familydefault{\sfdefault}
\tcbset{
    left = 1cm,
    right = -2mm,
    top = 0pt,
    bottom = 0pt,
    boxsep = 1ex,
    on line,
    arc = 3pt
}
\sisetup{text-family-to-math}

\newcommand\decformat[1]{\num[round-precision=2, round-mode=places]{#1}}
\newcommand\intformat[1]{(rank: \num[minimum-integer-digits=2]{#1})}

\newcommand\agecoeff[1]{\fpeval{(#1 - \minage)/(\maxage - \minage)}}
\newcommand\scorecoeff[1]{\fpeval{(#1 - \minscore)/(\maxscore - \minscore)}}

\newcommand\agebox[1]{%
    \newcommand\intcoeff{\fpeval{\agecoeff{#1}*100}}%
    \tcbox[tikznode, colback=orange!\intcoeff, colframe=orange!\intcoeff]{#1}
}

\newcommand\scorebox[1]{%
    \newcommand\scorewidth{\fpeval{\scorecoeff{#1}*4}cm}
    \tcbox[tikznode, colback=pink, colframe=pink, left=\scorewidth, right=-2mm]{#1}
}

\colorlet{hlcolor}{green!60!teal}
\colorlet{nocolor}{red!60!purple}
\renewcommand\check{{\Large\usym{2714}}}
\newcommand\cross{{\Large\usym{2718}}}

\newcommand\highlight[1]{\textcolor{hlcolor}{#1}}
\newcommand\gradehighlight[1]{%
    \if#1A%
        \highlight{#1}%
    \else #1%
    \fi
}

\newcommand\yesorno[1]{%
    \newcommand\yes{Yes}%
    \newcommand\no{No}%
    \newcommand\target{#1}%
    \ifx\yes\target
        \color{hlcolor}\check\ \target%
    \else
        \ifx\target\no
            \color{nocolor}\cross\ \target%
        \else
            \target%
        \fi
    \fi
}

\newcounter{rowofrank}
\setcounter{rowofrank}{1}
\newcommand\detectpodium[1]{%
    \ifnum#1=1%
        \xdef\rownumberone{\therowofrank}
    \fi
    \ifnum#1=2%
        \xdef\rownumbertwo{\therowofrank}
    \fi
    \ifnum#1=3%
        \xdef\rownumberthree{\therowofrank}
    \fi
}

\begin{document}

% Data retrieval from external .csv file. This is needed in particular to 
% compute minimum and maximum values of the columns of interest. We compute 
% those values by iterating along the columns and update a (previously defined) 
% macro. So we define the macros in a portion of code in a scope that's visible 
% before reading the data. We leverage the ternary construction "x ? y : z" of 
% the mathematical library of pgf/tikz for the update.
\csvreader[before reading=%
	\newcommand\maxage{0}%
	\newcommand\minage{10000}%
	\newcommand\maxscore{0}%
	\newcommand\minscore{10000},%
	separator=semicolon]%  we need to specify it explicitly  
	{data.csv}{age=\age, testscore=\testscore, rank=\rank}{%
	\pgfmathsetmacro{\minage}{(\age <= \minage) ? \age : \minage}%
	\pgfmathsetmacro{\maxage}{(\age >= \maxage) ? \age : \maxage}%
	\pgfmathsetmacro{\minscore}{%
		(\testscore <= \minscore) ? \testscore : \minscore}%
	\pgfmathsetmacro{\maxscore}{%
		(\testscore >= \maxscore) ? \testscore : \maxscore}%
	% We need to detect the row number of the highest three ranked elements, we
	% do so stepping a counter and saving it if the rank is not less than three.
	\stepcounter{rowofrank}\detectpodium{\rank}
}

\csvreader[head to column names,
	separator=semicolon,
	collect data,
	table head={id & name & age & grade & test1\_score & test2\_score & 
	final\_score & & registered\\},
	centered tabularray =
	{hlines={gray!50},
	row{1} = {font=\bfseries},
	rowsep = 2mm,
	colsep = 2mm,
	columns = {r, valign=h},
	cell{2-Z}{3} = {cmd=\agebox, valign=h},
	cell{2-Z}{4} = {cmd=\gradehighlight},
	cell{2-Z}{5} = {cmd=\scorebox, wd=5cm},
	cell{2-Z}{6} = {cmd=\scorebox, wd=4cm},
	cell{2-Z}{7} = {l, cmd=\decformat},
	cell{1}{7} = {r=1, c=2}{r},
	column{8} = {leftsep=-2pt},
	cell{2-Z}{8} = {cmd=\intformat},
	cell{2-Z}{9} = {cmd=\yesorno},
	cell{\rownumberone, \rownumbertwo, \rownumberthree}{7} = {cmd=\decformat, 
	fg=hlcolor},
	cell{\rownumberone, \rownumbertwo, \rownumberthree}{8} = {cmd=\intformat, 
	fg=hlcolor}
	}
]{data.csv}{rank=\rank}{%
\id & \name & \age & \grade & \testscore & \Testscore & \finalscore & \rank & 
\registered
}

\end{document}
