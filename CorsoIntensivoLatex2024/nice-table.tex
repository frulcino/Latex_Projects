\documentclass{article}

% We need to change the font of the table, we also need to change it into 
% something without serifs
\usepackage{DejaVuSans}
% Package to set output encoding to Type 1 font
\usepackage[T1]{fontenc}
% Main package for tables
\usepackage{tabularray}
% Package to retrieve external data from .csv files
\usepackage{csvsimple-l3}
% Package to load the 'tikznode' style of the tcolorbox and also the 
% mathematical library for computing minimum and maximum element column-wise.
% Notice also that tikz loads xcolor
\usepackage{tikz}
% Main package for box formatting
\usepackage{tcolorbox}
% Main package for units of measurement and number formatting
\usepackage{siunitx}
% Main package fo Unicode-based dingbats
\usepackage{utfsym}
% Main package for page layout setup
\usepackage[a4paper, landscape, margin=1cm]{geometry}


% Set default family to sans serif (another possibility is to just write 
% \sffamily at the beginning of the document
\renewcommand\familydefault{\sfdefault}

% Set the standard parameters for our desired boxes
\tcbset{left = 1cm, % amount of space to the left of box content
		right = -2mm, % the target table has boxes somewhat tight
		top = 0pt,
		bottom = 0pt,
		boxsep = 1ex, % font-dependent unit of measurement, roughly equal to 
					  % the height of a small 'x' in the current font
		on line, % we want the box content to share the same baseline of the 
				 % rest of the row
		arc = 3pt % curvature (radius) of the rounded corners
}

% The \num macro from siunitx typesets numbers in the math family default, so 
% we make it conform
\sisetup{text-family-to-math}

%%% Data-insensitive macros
% Ensures the desired format for decimals, i.e., 2 digits
\newcommand\decformat[1]{\num[round-precision=2, round-mode=places]{#1}}
% Ensures the desired format for integers, i.e., 2 digits, padding with zeros 
% if necessary, and adds the text '(rank: )'.
\newcommand\intformat[1]{(rank: \num[minimum-integer-digits=2]{#1})}

% Computes the coefficient coeff(x) = (x - a)/(b - a), a <= x <= b
% We use the powerful \fpeval here, but of course we could just have sticked 
% with \pgfmatsetmacro or something from the pgf math library.
% Even if not strictcly needed, we define two different macros for the 
% coefficients, one for ages and one for scores. Also notice that the only 
% scores that we are interested in are the ones in the fifth column, sparing us
% the tedious computations of the global minimum and maximum
\newcommand\agecoeff[1]{\fpeval{(#1 - \minage)/(\maxage - \minage)}}
\newcommand\scorecoeff[1]{\fpeval{(#1 - \minscore)/(\maxscore - \minscore)}}

% Inserts a box whose background color is proportional to its argument
\newcommand\agebox[1]{%
	\newcommand\intcoeff{\fpeval{\agecoeff{#1}*100}}%
	\tcbox[tikznode, colback=orange!\intcoeff,
		   colframe=orange!\intcoeff]{#1}}

% Inserts a box whose width is proportional to its argument
\newcommand\scorebox[1]{%
	\newcommand\scorewidth{\fpeval{\scorecoeff{#1}*4}cm}
	\tcbox[tikznode, colback=pink, colframe=pink, 
	left=\scorewidth, right=-2mm]{#1}}

% Custom colors and dingbats definitions
\colorlet{hlcolor}{green!60!teal}
\colorlet{nocolor}{red!60!purple}
\renewcommand\check{{\Large\usym{2714}}}
\newcommand\cross{{\Large\usym{2718}}}

% Custom definition of highlighting as coloring in our custom color.
\newcommand\highlight[1]{\textcolor{hlcolor}{#1}}

% If the grade is A we highlight it, otherwise we just typeset it.
\newcommand\gradehighlight[1]{%
	% We perform comparison using the \if conditional of TeX
	% It directly compares the two tokens: the first one is the argument '#1',
	% the second one is 'A'.
	\if#1A%
		\highlight{#1}%
	\else #1%
	\fi
}

% If the argument is 'Yes' it expands to it colored in green with the check 
%dingbat in front of it, if it is 'No', does the same but in red and with the 
%cross dingbat, otherwise it just expands to the argument.
\newcommand\yesorno[1]{%
	\newcommand\yes{Yes}%
	\newcommand\no{No}%
	\newcommand\target{#1}%
	% We compare things using \ifx, so we translate the argument into a macro 
	% in order to compares macros and not tokens
	\ifx\yes\target
		\color{hlcolor}\check\ \target%
	\else
		\ifx\target\no
			\color{nocolor}\cross\ \target%
		\else
			\target%
		\fi
	\fi}

% We create a counter GLOBALLY STEPPED
\newcounter{rowofrank}
\setcounter{rowofrank}{1} % Takes into account the header of the table is row 
% number 1!
% \xdef defines a GLOBAL macro, ALREADY EXPANDED to the textual representation 
% of the current value of the counter.
\newcommand\detectpodium[1]{%
	\ifnum#1=1%
		\xdef\rownumberone{\therowofrank}
	\fi
	\ifnum#1=2%
		\xdef\rownumbertwo{\therowofrank}
	\fi
	\ifnum#1=3%
		\xdef\rownumberthree{\therowofrank}
	\fi
}

\begin{document}

% Data retrieval from external .csv file. This is needed in particular to 
% compute minimum and maximum values of the columns of interest. We compute 
% those values by iterating along the columns and update a (previously defined) 
% macro. So we define the macros in a portion of code in a scope that's visible 
% before reading the data. We leverage the ternary construction "x ? y : z" of 
% the mathematical library of pgf/tikz for the update.
\csvreader[before reading=%
	\newcommand\maxage{0}%
	\newcommand\minage{10000}%
	\newcommand\maxscore{0}%
	\newcommand\minscore{10000},%
	separator=semicolon]%  we need to specify it explicitly  
	{data.csv}{age=\age, testscore=\testscore, rank=\rank}{%
	\pgfmathsetmacro{\minage}{(\age <= \minage) ? \age : \minage}%
	\pgfmathsetmacro{\maxage}{(\age >= \maxage) ? \age : \maxage}%
	\pgfmathsetmacro{\minscore}{%
		(\testscore <= \minscore) ? \testscore : \minscore}%
	\pgfmathsetmacro{\maxscore}{%
		(\testscore >= \maxscore) ? \testscore : \maxscore}%
	% We need to detect the row number of the highest three ranked elements, we
	% do so stepping a counter and saving it if the rank is not less than three.
	\stepcounter{rowofrank}\detectpodium{\rank}
}

\csvreader[head to column names,
	separator=semicolon,
	collect data,
	table head={id & name & age & grade & test1\_score & test2\_score & 
	final\_score & & registered\\},
	centered tabularray =
	{hlines={gray!50},
	row{1} = {font=\bfseries},
	rowsep = 2mm,
	colsep = 2mm,
	columns = {r, valign=h},
	cell{2-Z}{3} = {cmd=\agebox, valign=h},
	cell{2-Z}{4} = {cmd=\gradehighlight},
	cell{2-Z}{5} = {cmd=\scorebox, wd=5cm},
	cell{2-Z}{6} = {cmd=\scorebox, wd=4cm},
	cell{2-Z}{7} = {l, cmd=\decformat},
	cell{1}{7} = {r=1, c=2}{r},
	column{8} = {leftsep=-2pt},
	cell{2-Z}{8} = {cmd=\intformat},
	cell{2-Z}{9} = {cmd=\yesorno},
	cell{\rownumberone, \rownumbertwo, \rownumberthree}{7} = {cmd=\decformat, 
	fg=hlcolor},
	cell{\rownumberone, \rownumbertwo, \rownumberthree}{8} = {cmd=\intformat, 
	fg=hlcolor}
	}
]{data.csv}{rank=\rank}{%
\id & \name & \age & \grade & \testscore & \Testscore & \finalscore & \rank & 
\registered
}

\end{document}