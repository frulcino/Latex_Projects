\documentclass[]{report}
\usepackage{graphicx} %extended version of graphic package.
\usepackage{babel} %language stuff
\usepackage{caption} %caption package for not floats
\usepackage{csquotes} %language specific correct quoting
\usepackage{float} %force h,H ecc positioning for figures.
\usepackage{xfp} %calculations
\usepackage{blindtext}
\title{Note Corso Intensivo Latex}
\author{Gabor Riccardi}
\begin{document}

\maketitle
\chapter{Lezione 1}
\subsection{Quotes}
You can write accents just with the keyboad, no need for commands:
cliché \\
clich\'e \\
While to do quotes, don't do: "miao" or ''miao'', this gets displayed incorrently.
Instead, either use the package \emph{csquotes} and you can use <<miao>>?? \\ 
Or using latex low level code: \lq\lq miao '' or \lq\lq miao \rq \rq. But one drawback of this is the following: \lq\lq miao \rq \rq oh oh where the space after the right quote get's absorbed. We will look into this lates.
\subsection{Emphasis}
    The canon way to do it is \emph{miao}. The current consensus is to avoid the \emph{boldface}.
    You can also use textit, to be sure to use italics \textit{miao}. An other alternative \(\backslash\) it, but the latter is cursed, pls don't.
    
\subsection{Notes on Overleaf and Debugging}
    Overleaf compiles even with error, you shouldn't keep compiling if there is an error, because sooner or later they will pile up to something that doesn't compile.
    \\ Debugging, generally errors are trivial, but the message is not really helpful (maybe )

\subsection{Graphics}
Note: it's sufficient to do includegraphics, figure has nothing to do with graphics whatsoever. \\
\begin{figure}
    An example of some innocent text without graphics whatsoever
\end{figure}
\includegraphics{example-image}
\\ When you are trying to fix an image, if you have don't want to fix floats, don't use them, just don't use figure.
\\ See that you want to be center so you use \(\backslash\)begin\{center\} environment. Then you cannot use \(\backslash\)caption{} because this is for either \emph{figure} or \emph{table} environments.
\\ Note the standard for \(\backslash\)label{tab:meaningful description.}.
\\ So if we don't have the convenience of the floating environment, and do a caption? A package, \emph{caption}.\\ 
\begin{center}
    \includegraphics*{example-image}
    \captionof{figure}{Standard graphics}
    \label{img:ahah}
\end{center}

Something something more images together: \\

\includegraphics[height=2in]{example-image-a}
\includegraphics[height=2in]{example-image-b}
\\ Why is there white space, because we broke the line. \\
\includegraphics[height=2in]{example-image-a}% 
\includegraphics[height=2in]{example-image-b}
This way the breakline becomes commented by \%.

If you want the figure to be as far apart as possible:

\begin{center}
    \includegraphics[width=0.3\textwidth]{example-image-a}\hfill
    \includegraphics[width=0.3\textwidth]{example-image-b}
\end{center}
Not the actually use case scenario to use \(\backslash\)hfill but it works.
If you want them to be equally separated between eachother and the edges, we can add two more \lq\lq spring ''.
\begin{center}
    \hfill
    \includegraphics[width=0.3\textwidth]{example-image-a}\hfill
    \includegraphics[width=0.3\textwidth]{example-image-b} \hfill
\end{center}
This doesn't work well because reasons, but you can either use a \emph{minipage} environment. \\

\subsection{Tables}
Tabula is the standard environment for producing tabulas, it's 30 years old, or even more.
There is actually no inherent problem in using it but it is inconvenient for spacings, which are not nice. The consensus is that standard tabula tables are \emph{awful}.
There are now many other nice packages for doing it. I am tired, pisol.
Use the package \emph{Tabularray}! It's nice, you can also use imported csv files, instead of writing by hand, basically it's really good :D \\

\subsection{Calculation}
Floating point evaluation \fpeval{pi*5+7}

\end{document}