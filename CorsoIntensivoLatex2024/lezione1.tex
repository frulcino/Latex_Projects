\documentclass[11pt,a4paper,oneside,openany]{book}
%\usepackage[a4paperr, landscape, margin=1cm]{geometry} %for margins and page geometry
\usepackage{graphicx} %extended version of graphic package.
\usepackage{babel} %language stuff
\usepackage{caption} %caption package for not floats
% We need to change the font of the table, we also need to change it into 
% something without serifs
%\usepackage{DejaVuSans}
\usepackage{microtype}
\usepackage{amsmath}
\usepackage{amsthm} % for statements and proofs
\usepackage{mathtools} %it imports and enhances amsmath, highly adviced
% Package to set output encoding to Type 1 font
\usepackage[T1]{fontenc}
% Main package for tables
\usepackage{csquotes} %language specific correct quoting
\usepackage{float} %force h,H ecc positioning for figures.
\usepackage{xfp} %calculations
\usepackage{blindtext}
\usepackage{tabularray} %basic tables creation
\usepackage{tcolorbox} %for fancy boxes
\usepackage{xcolor} % for (enhanced) coloring
\usepackage{csvsimple} % for importing data
\usepackage{tikz} %for making graphics and the mathematical library pgf
\usepackage{siunitx}
% Main package fo Unicode-based dingbats
\usepackage{utfsym}
% Main package for page layout setup
%\usepackage[a4paper, landscape, margin=1cm]{geometry}


\title{Note Corso Intensivo Latex}
\author{Gabor Riccardi}





%mathematics command:
\newcommand\normone[1]{|| #1 ||} %problem: | can change meaning in the code so it can break so we should do:
\newcommand\normtwo[1]{\vert\vert #1 \vert\vert} %or
\newcommand\normthree[1]{\lVert #1\rVert} %but what if the input is tall?
\newcommand\normfour[1]{\bigg\lVert #1\bigg\rVert} %but this put too much space as usual, we would like it also to be optional.
\DeclarePairedDelimiter{\norm}{\lVert}{\rVert}

\newcommand\innerprodone[2]{\langle #1,#2 \rangle} %again not optimal in case of different sizes, so
\DeclarePairedDelimiter{\angular}{\langle}{\rangle}
\newcommand\innerprod[2]{\angular*{#1,#2}}
%now that we have math tools, we have the super power:
\begin{document}

\maketitle
\chapter{Lezione 1}
\subsection{Quotes}
You can write accents just with the keyboad, no need for commands:
cliché \\
clich\'e \\
While to do quotes, don't do: "miao" or ''miao'', this gets displayed incorrently.
Instead, either use the package \emph{csquotes} and you can use <<miao>>?? \\ 
Or using latex low level code: \lq\lq miao '' or \lq\lq miao \rq \rq. But one drawback of this is the following: \lq\lq miao \rq \rq oh oh where the space after the right quote get's absorbed. We will look into this lates.
\subsection{Emphasis}
    The canon way to do it is \emph{miao}. The current consensus is to avoid the \emph{boldface}.
    You can also use textit, to be sure to use italics \textit{miao}. An other alternative \(\backslash\) it, but the latter is cursed, pls don't.
    
\subsection{Notes on Overleaf and Debugging}
    Overleaf compiles even with error, you shouldn't keep compiling if there is an error, because sooner or later they will pile up to something that doesn't compile.
    \\ Debugging, generally errors are trivial, but the message is not really helpful (maybe )

\subsection{Graphics}
Note: it's sufficient to do includegraphics, figure has nothing to do with graphics whatsoever. \\
\begin{figure}
    An example of some innocent text without graphics whatsoever
\end{figure}
\includegraphics{example-image}
\\ When you are trying to fix an image, if you have don't want to fix floats, don't use them, just don't use figure.
\\ See that you want to be center so you use \(\backslash\)begin\{center\} environment. Then you cannot use \(\backslash\)caption{} because this is for either \emph{figure} or \emph{table} environments.
\\ Note the standard for \(\backslash\)label{tab:meaningful description.}.
\\ So if we don't have the convenience of the floating environment, and do a caption? A package, \emph{caption}.\\ 
\begin{center}
    \includegraphics*{example-image}
    \captionof{figure}{Standard graphics}
    \label{img:ahah}
\end{center}

Something something more images together: \\

\includegraphics[height=2in]{example-image-a}
\includegraphics[height=2in]{example-image-b}
\\ Why is there white space, because we broke the line. \\
\includegraphics[height=2in]{example-image-a}% 
\includegraphics[height=2in]{example-image-b}
This way the breakline becomes commented by \%.

If you want the figure to be as far apart as possible:

\begin{center}
    \includegraphics[width=0.3\textwidth]{example-image-a}\hfill
    \includegraphics[width=0.3\textwidth]{example-image-b}
\end{center}
Not the actually use case scenario to use \(\backslash\)hfill but it works.
If you want them to be equally separated between eachother and the edges, we can add two more \lq\lq spring ''.
\begin{center}
    \hfill
    \includegraphics[width=0.3\textwidth]{example-image-a}\hfill
    \includegraphics[width=0.3\textwidth]{example-image-b} \hfill
\end{center}
This doesn't work well because reasons, but you can either use a \emph{minipage} environment. \\

\subsection{Tables}
Tabula is the standard environment for producing tabulas, it's 30 years old, or even more.
There is actually no inherent problem in using it but it is inconvenient for spacings, which are not nice. The consensus is that standard tabula tables are \emph{awful}.
There are now many other nice packages for doing it. I am tired, pisol.
Use the package \emph{Tabularray}! It's nice, you can also use imported csv files, instead of writing by hand, basically it's really good :D \\

\subsection{Calculation}
Floating point evaluation \fpeval{pi*5+7}


\chapter{Lezione 2}
\section{Table creation}
Consensus among typographer says to not use vertical or horizontal separators, but if you need you can use \emph{vlines, hlines} in the options.

\begin{tblr}{}
    id & name \\
    1 & Bob \\
    2 & Ashley
\end{tblr}

\subsection{Import Data}
We are trying to reconstruct the example table.
We need to compute the maximum of certain columns, how to?
The best way to have the data on a csv file and then compute from where. A nice package to import table from this kind of files is \emph{csvsimple}.
FYI you already have the documentation of the packages on your latex distribution by typing \emph{texdoc <name package>} in the terminal, for examples \emph{texdoc tabularray}. \textbf{Ma che figata!}.

\subsection{Example of Macro creation}
\newcommand\C{\mathcal{C}}

The set of complex numbers \(\C\)

\subsection{Hardcore Math Library}
An hardcore math package, at the base of tikz is called \emph{pgf} mathematical library.


Difference between def and renewcommand is that def checks whether the command is already present or not.
Because you could break you code with this simple code: 
\(\backslash\)def\(\backslash\)relax\{TO read a nice book ahah\}
While renewcommand is somewhat protected, and want let you do it. An other reason to use macros is for changes in notations, to not have it to change all of them by hand.

\subsection{Maximum from file}
For the maximum we need conditionals. In Tex there are various conditionals already implemented.
Do not use \lq\lq\(\_ \)'' in the file name of tex files. It can break easily. \\ It is useful to debug stuff, one useful command is \emph{csvautotabulr}, it did't recognize ";" as a separator. We just need to specify the separator.



\chapter{Lesson 3}
\section{Mathematical!}
Interesting: 
\[
\text{Proof of a potato. This puts the square here} \hfill \qed    
\]
Look closely at the Difference
\begin{Huge}
\[
\cos(x)  \quad \cos \left( x \right)   
\]
\end{Huge}
Let's define a new command:
\[
\normone{x} \quad \normtwo{x} \quad \normthree{x}
\]
Problem:
\[
    \normthree{\frac12}\]
Solution:
\[
    \norm*{\frac12}\]

You cannot us ctrl f to search and replace the following:
\[
    a \cdot b \iff \langle a,b \rangle\]


The best thing to do would be to do a macro!
\[
    \innerprod{\frac12}{2}
    \]
Now let's consider the following spacing problem:
\[
    \sum_{0 \le i,j,k,h \le N} i^2 = \pi \quad \int_{x=1}^{x \to \infty}f(x,y)\,dx\]
This can be addressed really easily:
\[
    \sum_{\mathclap{0 \le i,j,k,h \le N}} i^2 = \pi \quad \int_{x=1}^{\mathrlap{x \to \infty}}f(x,y)\,dx
\]

I am an ino. Use substack for multiple conditions in the sum.
Now we want to put a list in a proof:
\begin{proof}
\begin{enumerate}
        \item The very first thing
        \item The very second thing
        \item The very third thing
    \end{enumerate}
\end{proof}
Horrible, just horrible. The problem here is that list tries to start on vertical mode. Proof also starts on vertical mode, so latex immediately tries to type set the vertical list. 
\begin{proof}\
    \begin{enumerate}
        \item The very first thing
        \item The very second thing
        \item The very third thing
    \end{enumerate}
\end{proof}
or
\begin{proof}\leavevmode
    \begin{enumerate}
        \item The very first thing
        \item The very second thing
        \item The very third thing
    \end{enumerate}
\end{proof}
The leavemode macro can be used whenever bad looking things are due to horizzontal vertical modes interaction.

\subsection{Useful or Cute Material}
\begin{itemize}
    \item \emph{CTAN} Comprehensive TEX Archive Network
    \item \emph{Latex2e} unofficial reference manual
    \item \emph{Tick-page} (pdf on CTAN) for decorating pages in latex
    \item \emph{Tickz-ext} to get a better tikz
    \item \emph{tikzpeople} they designed cute people in tikz
    \item \emph{Tikzpingus} for penguins
    \item \emph{tikzcd} for commutative diagrams
    \item \emph{dimline} units of measurements in graphics
    \item \emph{adigraph} for networks or graphics, more costumizable than tikzcd
    \item \emph{bbox} for covering boxes done right
    \item \emph{microtype} For microtypography, for microlocal adjustments of you document, you can enhance your document by simply importing it.
    \item 
\end{itemize}
\end{document}