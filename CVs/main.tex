%% start of file `template.tex'.
%% Copyright 2006-2015 Xavier Danaux (xdanaux@gmail.com).
%
% This work may be distributed and/or modified under the
% conditions of the LaTeX Project Public License version 1.3c,
% available at http://www.latex-project.org/lppl/.


\documentclass[11pt,a4paper,sans]{moderncv}        % possible options include font size ('10pt', '11pt' and '12pt'), paper size ('a4paper', 'letterpaper', 'a5paper', 'legalpaper', 'executivepaper' and 'landscape') and font family ('sans' and 'roman')

% moderncv themes
\moderncvstyle{classic}                             % style options are 'casual' (default), 'classic', 'banking', 'oldstyle' and 'fancy'
\moderncvcolor{blue}                               % color options 'black', 'blue' (default), 'burgundy', 'green', 'grey', 'orange', 'purple' and 'red'
%\renewcommand{\familydefault}{\sfdefault}         % to set the default font; use '\sfdefault' for the default sans serif font, '\rmdefault' for the default roman one, or any tex font name
%\nopagenumbers{}                                  % uncomment to suppress automatic page numbering for CVs longer than one page

% character encoding
\usepackage{verbatim}
\usepackage{fullpage}
\usepackage[utf8]{inputenc}    
\usepackage{tabularx}% if you are not using xelatex ou lualatex, replace by the encoding you are using
%\usepackage{CJKutf8}                              % if you need to use CJK to typeset your resume in Chinese, Japanese or Korean

% adjust the page margins
\usepackage[scale=0.75]{geometry}
%\setlength{\hintscolumnwidth}{3cm}                % if you want to change the width of the column with the dates
%\setlength{\makecvtitlenamewidth}{10cm}           % for the 'classic' style, if you want to force the width allocated to your name and avoid line breaks. be careful though, the length is normally calculated to avoid any overlap with your personal info; use this at your own typographical risks...

% personal data
\name{Gabor}{Riccardi}
\address{Via Alessandro Volta 3 }{Pavia (PV), 27100 Italia}{Data di nascita: 25 Aprile 1999}
\mobile{(+39) 377 1822565}
\email{gabor.riccardi01@universitadipavia.it}
\makeatletter\renewcommand*{\bibliographyitemlabel}{\@biblabel{\arabic{enumiv}}}\makeatother
%   to redefine the bibliography heading string ("Publications")
%\renewcommand{\refname}{Articles}

% bibliography with mutiple entries
%\usepackage{multibib}
%\newcites{book,misc}{{Books},{Others}}
%----------------------------------------------------------------------------------
%            content
%----------------------------------------------------------------------------------
\begin{document}
%\begin{CJK*}{UTF8}{gbsn}                          % to typeset your resume in Chinese using CJK
%-----       resume       ---------------------------------------------------------
\makecvtitle

\section{Istruzione}

\cventry{Ott 2023 - In corso}{Dottorato di Ricerca in Matematica Applicata}{Università di Pavia}{Pavia (PV)}{}{Sono attualmente Dottorando presso il programma in Computational Mathematics and Decision Sciences.}

\cventry{Feb 2022 \\- Set 2023}{Laurea Magistrale in Matematica}{Università di Pavia}{Pavia (PV)}{}{110/110 Cum Laude. Il Dipartimento di Matematica “F. Casorati” risulta tra gli 11 Dipartimenti di Eccellenza nazionali selezionati dal Governo nell’area delle scienze matematiche ed informatiche.}

\cventry{Ott 2019 \\- Feb 2022}{Laurea Triennale in Matematica}{Università di Pavia}{Pavia (PV)}{}{110/110 Cum Laude. Titolo della tesi:"On the Primitivity of the group
generated by the round functions of Iterated Block Ciphers and of PRESENT."}
\vspace{2pt}

\cventry{Set 2022 \\- Dic 2022}{Erasmus presso Master de Mathématiques Appliquées, Statistique }{Università di Rennes 2}{Rennes (RN)}{}{}

\cventry{Ott 2018 \\- In corso}{Alunno presso il Collegio Ghislieri}{Collegio Ghislieri}{Pavia (PV)}{}{Il Collegio Ghislieri è un collegio di merito riconosciuto dal MIUR. Ogni anno il collegio selezione 30 studenti mediante una competizione nazione e offre alloggio e corsi extra universitari.}
\vspace{2pt}

\cventry{Set 2013 \\- Lug 2018}{Diploma di Perito Chimico}{Istituto Tecnico Industriale Statale G.Cardano}{Pavia (PV)}{}{Diploma di Maturità di Perito Chimico, Valutazione: 100/100 e Lode}

\cventry{Ott 2018 \\- Ott 2019}{Primo anno Laurea Triennale in Chimica}{Pavia (PV)}{}{Sono stato studente presso la Facoltà di Chimica a Pavia, completando il piano di studi del primo anno, prima di passare alla Facoltà di Matematica}{}




\section{Capacità comunicative}

\cventry{Apr 2018 \\ - In corso}{Organizzazione e divulgazione al pubblico}{Indiscienza e Mathpetizer}{}{Ghislieri Scienza}{Ghislieri scienza è una associazione universitaria dedicata alla divulgazione scientifica. In questa associazione ricopro vari ruoli. Sono membro del \emph{direttivo} dell'associazione. Sono \emph{Responsabile} dell'organizzazione di Indiscienza, un festival di divulgazione scientifica. Ricopro il ruolo di \emph{Responsabile} per la comunicazione con le Scuole e \emph{Responsabile} per la Stampante 3D.{}{}{ Sono inoltre \emph{Organizzatore} dell'evento di divulgazione matematica Mathpetizer. Il suo scopo è avvicinare persone di tutte le fasce di età dalla matematica mediante la matematica ricreativa.}}
\vspace{2pt}

\cventry{2014 \\ -2017}{Laboratorio di chimica Grest}{}{Animatore e Organizzatore del laboratorio di esperienze di Chimica a Pavia}{}{}
%laboratorio di chimica grest

%\begin{center}
%\begin{tabular}{|c | c| c|} 
% \hline \emph{Anno di Corso 1}  & \emph{Anno Scolastico 2019/2020} & $n^\circ$ cfu
% \\ 
% \hline
% Algebra Lineare & 30 & 9\\
% \hline
% Analisi Matematica 1 & 30 & 9 \\ \hline
% Analisi Matematica 2 & 30 e lode &9 \\ \hline
% Fisica Generale 1 & 27&9 \\ \hline
% Lingua Inglese & Idoneo &3\\ \hline
% Programmazione 1 & 25& 6 \\ \hline
% Programmazione 2 & Idoneo & 3\\ \hline \hline
% \emph{Anno di Corso 2} & \emph{Anno Scolastico 2020/2021} & %\\ \hline
% Algebra 1 & 30 &9\\ \hline
% Analisi Matematica 3 & 29&9 \\ \hline
% Analisi Numerica & 28&12 \\ \hline
% Elementi di Probabilità & 27&9 \\ \hline
% Fondamenti di Meccanica & 29&9 \\ \hline
% Geometria 2 & 27&9 \\ \hline
% Algebra 2 & 30&6 \\ \hline
% Analisi Matematica 4 & 30&9 \\ \hline
% Equazioni della Fisica Matematica & 30 e lode&6 \\ \hline
% Fondamenti della Matematica & 30&6 \\ \hline
% Chimica Organica e Laboratorio & 30 & 15 \\ \hline
% Elementi di Statistica & 30 e lode (non ancora registrato)&6 5\\ \hline
% Complementi di Geometria & 28 (non ancora registrato)&6 \\ %\hline
% Fisica 2 & 30 (non ancora registrato)&9 \\ \hline
%\end{tabular}
%\end{center}
%\caption{Tabella contente i voti degli esami sostenuti presso la Facoltà di Matematica a Pavia}
\newpage
\section{Esperienze lavorative}
\cventry{Ott 2023 - Mar 2024}{Tirocinio presso il Joint Research Centre}{Commissione Europea}{Ispra (VA)}{}{Il JRC fornisce un sostegno scientifico e tecnico alla progettazione, allo sviluppo, all'attuazione e al controllo delle politiche dell'Unione europea. Ho lavorato presso l'unità Enery and Power Systems sulla implementazione di modelli della rete elettrica Europea ai fini dello studio della resilienza in un contesto di crescente stocasticità dovuta all'implementazione di fonti rinnovabili. }

\cventry{Ago 2022 \\- Set 2022}{Internship di ricerca con borsa}{Università di Monaco}{}{}{Ho lavorato presso il laboratorio di Prof.ssa Antonia Wachter-zeh sulle metriche proiettive nella crittografia.}

\cventry{2021 \\- In corso}{Tutor}{Università di Pavia}{}{}{Sono correntemente tutor presso la facoltà di ingegneria e la facoltà di matematica presso UniPv nelle seguenti materie: Algebra Lineare, Meccanica Razionale e Geometria.}
\vspace{2pt}


\section{Premi e riconoscimenti}

\cventry{Set 2013 \\- Mag 2018}{Olimpiadi della Chimica}{}{}{}{Mi sono classificato primo alle gare regionali a Milano alle Olimpiadi della Chimica per la categoria A l'anno 2015  e sono arrivato 16simo alle finali nazionali a Frascati lo stesso anno. Classificato 8avo alle \emph{Gare di chimica nazionali} a Chieti l'anno 2017. Classificato primo nelle gare regionali a Milano alle \emph{Olimpiadi di Chimica} per la categoria C l'anno 2018 }

\cventry{Giu 2017}{Piano Nazionale Lauree Scientifiche - Chimica 2014-2016}{}{}{}{Sono risultato vincitore di uno dei premi messi a disposizione nell'ambito del Piano Nazionale Lauree Scientifiche - Chimica 2014-2016 per le relazioni sulle attività di stage svolte presso il Dipartimento di Chimica dell'Ateneo di Pavia l'anno 2017-2017 }

%\cventry{2004 \\ 2011}{Riconoscimenti sportivi}{}{}{}{Ho fatto l'agonismo di pattinaggio artistico a rotelle per 7 anni, classificandomi al decimo posto alle nazionali.}



\section{Competenze linguistiche}
\cvitem{Italiano}{Madrelingua}
\cvitem{Ungherese}{Madrelingua}

\begin{center}
    \begin{tabular}{c c c c c c }
    \hline
                 &  Understanding & & Speaking & \hfill & Writing \\ \hline
                 &  Listening & Reading &   & Spoken Production & \\ \hline
        Inglese (First Certificate): & C1 (187)  & C1 (189) & C1 (186) & B2 (177) & B2 (168) \\ \hline
        Francese (DELF certificate):  & B1 & B1 & B1 & B1 & B1 \\ \hline
                
    \end{tabular}
\end{center}


\cvitem{Esperanto}{Certificato I.I.E. A1 ($1^{\circ}$ grado)}

\section{Competenze informatiche}
\cvitem{}{Conoscenza elevata di: Python (Pyomo, Numpy, Pandas, Sage), Latex, R}

\cvitem{}{Conoscenza intermedia di: SQL, Matlab}

\cvitem{}{Nuova Ecdl}

\newline 
\newline 
Autorizzo il trattamento dei miei dati personali presenti nel curriculum vitae ai sensi del Decreto Legislativo 30 giugno 2003, n. 196 e del GDPR (Regolamento UE
2016/679).

\section{Sport e Hobby}

\cvitem{}{Pratico regolarmente la corsa e boxing presso la Polisportiva Pavese }

\cvitem{}{Suono il piano e l'ukulele}

\cvitem{}{Mi piace organizzare e  partecipare a eventi di divulgazione scientifica per persone di tutte le età}

\newpage

\section {Programma degli studi e ricerche per l'anno di perfezionamento}

Gli studi Optimal Power Flow sono essenziali per analizzare il comportamento delle reti elettriche in diverse condizioni di carico. Tuttavia, il problema dell'Optimal Power FLow (OPF), che cerca di ottimizzare i costi di generazione, le perdite, le emissioni e le violazioni dei vincoli, è altamente complesso, non liscio, non convesso e non lineare. Inoltre, la struttura del 
 grafo intrinseca della rete pone sfide nella comprensione della sua influenza sul problema dell'OPF. Il problema diventa ancora più complesso poiché gli OPF devono adattarsi a nuove variabili stocastiche legate alla produzione e al consumo di energia, questa classe di problemi è chiamata OPF stocastico. Durante l'anno di perfezionamento ho l'obiettivo di (1) sviluppare una comprensione più approfondita della relazione tra la struttura del grafo della rete e il problema dell'OPF, concentrandosi sullo studio di una nuova famiglia di modelli simili a Jabr, e (2) sviluppare e implementare un modello accurato di OPF stocastico distribuzionalmente robusto su una porzione della rete di trasmissione della Sicilia, utilizzando dati provenienti da CESI.
\nocite{*}
\bibliographystyle{unsrt}
\bibliography{publications}                      % 'publications' is the name of a BibTeX file

% Publications from a BibTeX file using the multibib package
%\section{Publications}
%\nocitebook{book1,book2}
%\bibliographystylebook{plain}
%\bibliographybook{publications}                   % 'publications' is the name of a BibTeX file
%\nocitemisc{misc1,misc2,misc3}
%\bibliographystylemisc{plain}
%\bibliographymisc{publications}                   % 'publications' is the name of a BibTeX file

%\clearpage
%-----       letter       ---------------------------------------------------------
% recipient data
% \recipient{Company Recruitment team}{Company, Inc.\\123 somestreet\\some city}
% \date{January 01, 1984}
% \opening{Dear Sir or Madam,}
% \closing{Yours faithfully,}
% \enclosure[Attached]{curriculum vit\ae{}}          % use an optional argument to use a string other than "Enclosure", or redefine \enclname
% \makelettertitle

% Albert Einstein discovered that $e=mc^2$ in 1905.

%\[ e=\lim_{n \to \infty} \left(1+\frac{1}{n}\right)^n \]

% \makeletterclosing

% %\clearpage\end{CJK*}                              % if you are typesetting your resume in Chinese using CJK; the \clearpage is required for fancyhdr to work correctly with CJK, though it kills the page numbering by making \lastpage undefined
\end{document}


%% end of file `template.tex'.



