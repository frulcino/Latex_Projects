   
\section{Structure-Preserving Aggregations}

\begin{frame}{General definition of Structure Preserving Aggregation}

\begin{definition}
    \label{def:structure preserving aggregation}
    Given an LP, a row and column aggregation with respect to partitions \(\rowP,  \colP\) is said to be \myemph{structure-preserving} if \(f:[n] \rightarrow [\tilde{n}]\)
    %
    \pause, given by
    \[
    f : c \mapsto C \quad \text{ where } C \text{ is the element in } \colP \text{ such that } c\in C,
    \]
    \pause is such that for each \(R \in \rowP_{>1}\) and all \(r \in R\):
    \[
    f|_{\supp(\Av_r)} : \supp(\Av_r) \rightarrow \supp(\At_R)
    \]
    is a bijection, and \pause
    \[
    \Av_{r,c} = \At_{R,f(c)} \quad \text{for all } c \in \supp(\Av_r)_{>1}.
    \]
  \end{definition}
\end{frame}

\begin{frame}{(2) When can we obtains a solution of \CEPT given a solution of the aggregated problem \CEPP? \\ \vspace{1cm}}
    Given a solution \(\xt\) of the aggregated problem \CEPP, we want to extend it to a solution \(\xv\) of \CEPT. \\
    \only<1>{ 
    What is a good candidate solution? We keep unaggregated variables the same, that is:
    \[
        \nhi \coloneqq \nhit,\; \nsi \coloneqq \nsit, \; \nsi \coloneqq \nsit
    \]}
    \uncover<2-5>{
    How do we define unaggregated variables? Remember that:
    \[
    \sum_{t \in \{1,\ldots,t_1\}} \HtEii{t}{}{} = \HtEii{\{1,\ldots,t_1\}}{}{}
    \]}
    \uncover<3-5>{
    So we define for all \(t \in \{t,\ldots,t_1\}\):
    \[
        \HtEii{t}{}{} = \rho_t\HtEii{\{1,\ldots,t_1\}}{}{}
    \]}
    \uncover<4-5>{
    Where \(\rho_t\) corresponds to \(\frac{\text{Net power at time t}}{\text{Total Net power at interval} \{1,\ldots,t_1\}}\).
    }
    \uncover<5>{
        \begin{proposition}
            Is such a solution \(\xv\) of \CEPT is well defined, and \(\rho_t \geq 0\), then it is optimal.
        \end{proposition}
    }
\end{frame}
\begin{frame}{Generalization to structure preserving aggregations}
Given a solution \(\xt\) of the aggregated problem \(\tilde{\text{LP}}\), we want to extend it to a solution \(\xv\) of LP. \\
We keep unaggregated vairiables the same, that is, if \(\{c\} \in \colP\), then \(\xv_c = \xt_c\). \\ 
Otherwise, for a given aggregated variable \(c \in C\) in the support of the row \(r\) we search solution of the form: 
\begin{align}
\xv_c \coloneqq \rhov_r \xt_{f(c)} \; 
\end{align}
Then:
    \begin{align}
        \Av_r \xv &=\Av_{r, \colP_{=1}} \xv_{\colP_{=1}} + \sum_{c \in \supp(\Av_r)_{>1}} \A_{r, c} \xv_{c} = \Av_{r, \colP_{=1}} \xv_{\colP_{=1}} + \sum_{c \in \supp(\Av_r)_{>1}} \At_{R, f(c)} \xv_{c}  \\
         &=  \Av_{r, \colP_{=1}} \xt_{\colP_{=1}} + \rhov_r \sum_{\col \in \supp(\At_R)_{>1}} \At_{R, \col}  \xt_{\col} = b_r
    \end{align}
    Since \(\sum_{\col \in \supp(\At_R)_{>1}} \At_{R, \col}  \xt_{\col} = \bt_r
    - \At_{R, \colP_{=1}} \xt_{\colP_{=1}}\), the equality holds if and only if:
    \begin{align}
        \rhov_r \coloneqq \frac{\bv_r - \Av_{r, \colP_{=1}} \xt_{\colP_{=1}}}{\bt_r
            - \At_{R, \colP_{=1}} \xt_{\colP_{=1}}}.
    \end{align}
\end{frame}

\begin{frame}{Generalization to structure preserving aggregations}
    \begin{proposition}
        \label{ob:aggrconstr}
       Let \(\xt\) be a solution to the aggregated problem, define 
        \[\rhov_r \coloneqq \frac{\bv_r - \Av_{r, \colP_{=1}} \xt_{\colP_{=1}}}{\bt_r
            - \At_{R, \colP_{=1}} \xt_{\colP_{=1}}}.\]
        If \(\rhov_r \geq 0\) and \(\xv \in \bR^n\) satisfies \(\xv_{\colP_{=1}} = \xt_{\colP_{=1}}\) and \(\xv_{c} = \rhov_r \xt_{f(c)}\) for all \(c \in \supp(\Av_r)_{>1}\), then \(\xv\)  satisfies the constraints \(\Av_r \xv = \bv_r\) and \(\xv_{\supp(\Av_r)} \geq 0\) of the original problem.
    \end{proposition}
    
    Whenever such a solution is well defined we refer to it as a \(\rho\)-solution.
\end{frame}
% \begin{frame}{rho solutions}

%     \begin{definition}
%         Let \(\rhov_r\) be defined as in Observation \ref{ob:aggrconstr} for all \(r \in R \in \rowP_{>1}\). Let \(\xv \in \bR^n\) be defined as \(\xv_{\colP_{=1}} \coloneqq \xt_{\colP_{=1}}\) and \(\xv_{c} \coloneqq \rhov_r \xt_{f(c)}\) for all  \(r \in R \in \rowP_{>1}\) and \(c \in \supp(\Av_r)\). 
%         %
%         Then, \(\xv\)  is well defined if and uncover if for all \(r, r' \in R \in \rowP_{>1}\) such that \(\supp(\Av_r) \cap \supp(\Av_{r'})\neq 0\), we have \(\rhov_r = \rhov_{r'}\). 
%         %
%         If \(\rhov \geq 0\), we refer to \(\xv\)  as a \(\rhov\)-solution. 
%       \end{definition}
% \end{frame}


\begin{frame}{(3) Can we effectively iterate over finer time partitions to obtain a solution to \CEPT?}
\begin{enumerate}
    \item Impose the constraints relative to an initial time partition and solve the corresponding LP.
    %
    \item Select a time interval such that either \(\rho_r < 0\) or the \(\rho\)-solution is not well defined and refine it into smaller sub-intervals.
    %
    \item Add the constraints relative to each sub-interval of the selected interval. Solve the model again but using a warm-start.
    %
    \item Repeat steps 2 and 3 until a specified halting condition is met.
    \end{enumerate}
    \vspace{1cm}
    Halting condition:
    \begin{enumerate}
        \item \(\rhov_r\) is constant over the hypergraph associated to the aggregated problem \CEPP and  and \(\rhov \geq 0\). \label{eq: condition 2 iter}
        \item A maximum number of iterations is reached
    \end{enumerate}
    
    \begin{observation}
        If the algorithm alts before the second condition is met, then the \(\rho\)-solution is well defined and is an optimal solution for \CEPT.
    \end{observation}
\end{frame}

\section{Results}

\begin{frame}{Computational Results}
    \begin{itemize}
        \item 5-node network simulation
        \item Comparison of random vs heuristic-based refinement
        \item Faster convergence with structure-preserving methods
    \end{itemize}


\begin{figure}[t!]
    \centering
    \begin{tikzpicture}[scale=0.9]
    \begin{axis}[
        xlabel={Iteration number},
        ylabel={Cost (Millions of Euros)},
        legend pos=north west,
        legend style={draw=none},
        grid=major,
        thick,
        cycle list name=color list,
        yticklabel style={
                /pgf/number format/fixed,        % Use fixed-point notation
                /pgf/number format/precision=4,  % Set precision to 4 decimal places
                /pgf/number format/fixed zerofill % Add trailing zeros if necessary
            }]
    % Importing rho data
    \addplot+[
        thick,
        mark=none,
        color=blue
    ] table[x index=0, y index=1, col sep=space] {plots/costs_rho.csv};
    % Importing random data
    \addplot+[
        thick,
        mark=none,
        color=green
    ] table[x index=0, y index=1, col sep=space] {plots/costs_random.csv};
    % Legend
    \legend{rho, random}
    \end{axis}
    \end{tikzpicture}
    \caption{Cost over iterations of rho selection method versus random selection method}
      \label{fig:rho_vs_average}
\end{figure}
    
\end{frame}

\begin{frame}

    \begin{figure}[t!]
        \centering
        \begin{tikzpicture}[scale=0.9]
        \begin{axis}[
            xlabel={Iteration number},
            ylabel={Total capacity (Kg)},
            legend pos=south east,
            legend style={draw=none},
            grid=major,
            thick,
            cycle list name=color list,
            yticklabel={\pgfmathprintnumber[fixed, precision=5]{\tick}} 
        ]
        
        \addplot+[
            thick,
            mark=none,
            color=blue
        ] table[x index=0, y index=1, col sep=space] {plots/mhte_rho.csv};
        
        \addplot+[
            thick,
            mark=none,
            color=green
        ] table[x index=0, y index=1, col sep=space] {plots/mhte_random.csv};
        
        \legend{rho, random}
        \end{axis}
        \end{tikzpicture}
        \caption{Hydrogen to Power Capacity over iterations}
          \label{fig:mhte_over_iter}
        \end{figure}
        
\end{frame}
\begin{frame}{Conclusion}
    \begin{itemize}
        \item Time series aggregation reduces computational costs
        \item Preserves structure and accuracy
        \item The index \(\rho\) can also be interpreted as a fractional net power production index.
        \item Future direction: Is there always a non trivial time aggregation which induces a tight relaxation of \CEPT?
    \end{itemize}
\end{frame}